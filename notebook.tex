
% Default to the notebook output style

    


% Inherit from the specified cell style.




    
\documentclass[11pt]{article}

    
    
    \usepackage[T1]{fontenc}
    % Nicer default font (+ math font) than Computer Modern for most use cases
    \usepackage{mathpazo}

    % Basic figure setup, for now with no caption control since it's done
    % automatically by Pandoc (which extracts ![](path) syntax from Markdown).
    \usepackage{graphicx}
    % We will generate all images so they have a width \maxwidth. This means
    % that they will get their normal width if they fit onto the page, but
    % are scaled down if they would overflow the margins.
    \makeatletter
    \def\maxwidth{\ifdim\Gin@nat@width>\linewidth\linewidth
    \else\Gin@nat@width\fi}
    \makeatother
    \let\Oldincludegraphics\includegraphics
    % Set max figure width to be 80% of text width, for now hardcoded.
    \renewcommand{\includegraphics}[1]{\Oldincludegraphics[width=.8\maxwidth]{#1}}
    % Ensure that by default, figures have no caption (until we provide a
    % proper Figure object with a Caption API and a way to capture that
    % in the conversion process - todo).
    \usepackage{caption}
    \DeclareCaptionLabelFormat{nolabel}{}
    \captionsetup{labelformat=nolabel}

    \usepackage{adjustbox} % Used to constrain images to a maximum size 
    \usepackage{xcolor} % Allow colors to be defined
    \usepackage{enumerate} % Needed for markdown enumerations to work
    \usepackage{geometry} % Used to adjust the document margins
    \usepackage{amsmath} % Equations
    \usepackage{amssymb} % Equations
    \usepackage{textcomp} % defines textquotesingle
    % Hack from http://tex.stackexchange.com/a/47451/13684:
    \AtBeginDocument{%
        \def\PYZsq{\textquotesingle}% Upright quotes in Pygmentized code
    }
    \usepackage{upquote} % Upright quotes for verbatim code
    \usepackage{eurosym} % defines \euro
    \usepackage[mathletters]{ucs} % Extended unicode (utf-8) support
    \usepackage[utf8x]{inputenc} % Allow utf-8 characters in the tex document
    \usepackage{fancyvrb} % verbatim replacement that allows latex
    \usepackage{grffile} % extends the file name processing of package graphics 
                         % to support a larger range 
    % The hyperref package gives us a pdf with properly built
    % internal navigation ('pdf bookmarks' for the table of contents,
    % internal cross-reference links, web links for URLs, etc.)
    \usepackage{hyperref}
    \usepackage{longtable} % longtable support required by pandoc >1.10
    \usepackage{booktabs}  % table support for pandoc > 1.12.2
    \usepackage[inline]{enumitem} % IRkernel/repr support (it uses the enumerate* environment)
    \usepackage[normalem]{ulem} % ulem is needed to support strikethroughs (\sout)
                                % normalem makes italics be italics, not underlines
    

    
    
    % Colors for the hyperref package
    \definecolor{urlcolor}{rgb}{0,.145,.698}
    \definecolor{linkcolor}{rgb}{.71,0.21,0.01}
    \definecolor{citecolor}{rgb}{.12,.54,.11}

    % ANSI colors
    \definecolor{ansi-black}{HTML}{3E424D}
    \definecolor{ansi-black-intense}{HTML}{282C36}
    \definecolor{ansi-red}{HTML}{E75C58}
    \definecolor{ansi-red-intense}{HTML}{B22B31}
    \definecolor{ansi-green}{HTML}{00A250}
    \definecolor{ansi-green-intense}{HTML}{007427}
    \definecolor{ansi-yellow}{HTML}{DDB62B}
    \definecolor{ansi-yellow-intense}{HTML}{B27D12}
    \definecolor{ansi-blue}{HTML}{208FFB}
    \definecolor{ansi-blue-intense}{HTML}{0065CA}
    \definecolor{ansi-magenta}{HTML}{D160C4}
    \definecolor{ansi-magenta-intense}{HTML}{A03196}
    \definecolor{ansi-cyan}{HTML}{60C6C8}
    \definecolor{ansi-cyan-intense}{HTML}{258F8F}
    \definecolor{ansi-white}{HTML}{C5C1B4}
    \definecolor{ansi-white-intense}{HTML}{A1A6B2}

    % commands and environments needed by pandoc snippets
    % extracted from the output of `pandoc -s`
    \providecommand{\tightlist}{%
      \setlength{\itemsep}{0pt}\setlength{\parskip}{0pt}}
    \DefineVerbatimEnvironment{Highlighting}{Verbatim}{commandchars=\\\{\}}
    % Add ',fontsize=\small' for more characters per line
    \newenvironment{Shaded}{}{}
    \newcommand{\KeywordTok}[1]{\textcolor[rgb]{0.00,0.44,0.13}{\textbf{{#1}}}}
    \newcommand{\DataTypeTok}[1]{\textcolor[rgb]{0.56,0.13,0.00}{{#1}}}
    \newcommand{\DecValTok}[1]{\textcolor[rgb]{0.25,0.63,0.44}{{#1}}}
    \newcommand{\BaseNTok}[1]{\textcolor[rgb]{0.25,0.63,0.44}{{#1}}}
    \newcommand{\FloatTok}[1]{\textcolor[rgb]{0.25,0.63,0.44}{{#1}}}
    \newcommand{\CharTok}[1]{\textcolor[rgb]{0.25,0.44,0.63}{{#1}}}
    \newcommand{\StringTok}[1]{\textcolor[rgb]{0.25,0.44,0.63}{{#1}}}
    \newcommand{\CommentTok}[1]{\textcolor[rgb]{0.38,0.63,0.69}{\textit{{#1}}}}
    \newcommand{\OtherTok}[1]{\textcolor[rgb]{0.00,0.44,0.13}{{#1}}}
    \newcommand{\AlertTok}[1]{\textcolor[rgb]{1.00,0.00,0.00}{\textbf{{#1}}}}
    \newcommand{\FunctionTok}[1]{\textcolor[rgb]{0.02,0.16,0.49}{{#1}}}
    \newcommand{\RegionMarkerTok}[1]{{#1}}
    \newcommand{\ErrorTok}[1]{\textcolor[rgb]{1.00,0.00,0.00}{\textbf{{#1}}}}
    \newcommand{\NormalTok}[1]{{#1}}
    
    % Additional commands for more recent versions of Pandoc
    \newcommand{\ConstantTok}[1]{\textcolor[rgb]{0.53,0.00,0.00}{{#1}}}
    \newcommand{\SpecialCharTok}[1]{\textcolor[rgb]{0.25,0.44,0.63}{{#1}}}
    \newcommand{\VerbatimStringTok}[1]{\textcolor[rgb]{0.25,0.44,0.63}{{#1}}}
    \newcommand{\SpecialStringTok}[1]{\textcolor[rgb]{0.73,0.40,0.53}{{#1}}}
    \newcommand{\ImportTok}[1]{{#1}}
    \newcommand{\DocumentationTok}[1]{\textcolor[rgb]{0.73,0.13,0.13}{\textit{{#1}}}}
    \newcommand{\AnnotationTok}[1]{\textcolor[rgb]{0.38,0.63,0.69}{\textbf{\textit{{#1}}}}}
    \newcommand{\CommentVarTok}[1]{\textcolor[rgb]{0.38,0.63,0.69}{\textbf{\textit{{#1}}}}}
    \newcommand{\VariableTok}[1]{\textcolor[rgb]{0.10,0.09,0.49}{{#1}}}
    \newcommand{\ControlFlowTok}[1]{\textcolor[rgb]{0.00,0.44,0.13}{\textbf{{#1}}}}
    \newcommand{\OperatorTok}[1]{\textcolor[rgb]{0.40,0.40,0.40}{{#1}}}
    \newcommand{\BuiltInTok}[1]{{#1}}
    \newcommand{\ExtensionTok}[1]{{#1}}
    \newcommand{\PreprocessorTok}[1]{\textcolor[rgb]{0.74,0.48,0.00}{{#1}}}
    \newcommand{\AttributeTok}[1]{\textcolor[rgb]{0.49,0.56,0.16}{{#1}}}
    \newcommand{\InformationTok}[1]{\textcolor[rgb]{0.38,0.63,0.69}{\textbf{\textit{{#1}}}}}
    \newcommand{\WarningTok}[1]{\textcolor[rgb]{0.38,0.63,0.69}{\textbf{\textit{{#1}}}}}
    
    
    % Define a nice break command that doesn't care if a line doesn't already
    % exist.
    \def\br{\hspace*{\fill} \\* }
    % Math Jax compatability definitions
    \def\gt{>}
    \def\lt{<}
    % Document parameters
    \title{131 Project v2}
    
    
    

    % Pygments definitions
    
\makeatletter
\def\PY@reset{\let\PY@it=\relax \let\PY@bf=\relax%
    \let\PY@ul=\relax \let\PY@tc=\relax%
    \let\PY@bc=\relax \let\PY@ff=\relax}
\def\PY@tok#1{\csname PY@tok@#1\endcsname}
\def\PY@toks#1+{\ifx\relax#1\empty\else%
    \PY@tok{#1}\expandafter\PY@toks\fi}
\def\PY@do#1{\PY@bc{\PY@tc{\PY@ul{%
    \PY@it{\PY@bf{\PY@ff{#1}}}}}}}
\def\PY#1#2{\PY@reset\PY@toks#1+\relax+\PY@do{#2}}

\expandafter\def\csname PY@tok@w\endcsname{\def\PY@tc##1{\textcolor[rgb]{0.73,0.73,0.73}{##1}}}
\expandafter\def\csname PY@tok@c\endcsname{\let\PY@it=\textit\def\PY@tc##1{\textcolor[rgb]{0.25,0.50,0.50}{##1}}}
\expandafter\def\csname PY@tok@cp\endcsname{\def\PY@tc##1{\textcolor[rgb]{0.74,0.48,0.00}{##1}}}
\expandafter\def\csname PY@tok@k\endcsname{\let\PY@bf=\textbf\def\PY@tc##1{\textcolor[rgb]{0.00,0.50,0.00}{##1}}}
\expandafter\def\csname PY@tok@kp\endcsname{\def\PY@tc##1{\textcolor[rgb]{0.00,0.50,0.00}{##1}}}
\expandafter\def\csname PY@tok@kt\endcsname{\def\PY@tc##1{\textcolor[rgb]{0.69,0.00,0.25}{##1}}}
\expandafter\def\csname PY@tok@o\endcsname{\def\PY@tc##1{\textcolor[rgb]{0.40,0.40,0.40}{##1}}}
\expandafter\def\csname PY@tok@ow\endcsname{\let\PY@bf=\textbf\def\PY@tc##1{\textcolor[rgb]{0.67,0.13,1.00}{##1}}}
\expandafter\def\csname PY@tok@nb\endcsname{\def\PY@tc##1{\textcolor[rgb]{0.00,0.50,0.00}{##1}}}
\expandafter\def\csname PY@tok@nf\endcsname{\def\PY@tc##1{\textcolor[rgb]{0.00,0.00,1.00}{##1}}}
\expandafter\def\csname PY@tok@nc\endcsname{\let\PY@bf=\textbf\def\PY@tc##1{\textcolor[rgb]{0.00,0.00,1.00}{##1}}}
\expandafter\def\csname PY@tok@nn\endcsname{\let\PY@bf=\textbf\def\PY@tc##1{\textcolor[rgb]{0.00,0.00,1.00}{##1}}}
\expandafter\def\csname PY@tok@ne\endcsname{\let\PY@bf=\textbf\def\PY@tc##1{\textcolor[rgb]{0.82,0.25,0.23}{##1}}}
\expandafter\def\csname PY@tok@nv\endcsname{\def\PY@tc##1{\textcolor[rgb]{0.10,0.09,0.49}{##1}}}
\expandafter\def\csname PY@tok@no\endcsname{\def\PY@tc##1{\textcolor[rgb]{0.53,0.00,0.00}{##1}}}
\expandafter\def\csname PY@tok@nl\endcsname{\def\PY@tc##1{\textcolor[rgb]{0.63,0.63,0.00}{##1}}}
\expandafter\def\csname PY@tok@ni\endcsname{\let\PY@bf=\textbf\def\PY@tc##1{\textcolor[rgb]{0.60,0.60,0.60}{##1}}}
\expandafter\def\csname PY@tok@na\endcsname{\def\PY@tc##1{\textcolor[rgb]{0.49,0.56,0.16}{##1}}}
\expandafter\def\csname PY@tok@nt\endcsname{\let\PY@bf=\textbf\def\PY@tc##1{\textcolor[rgb]{0.00,0.50,0.00}{##1}}}
\expandafter\def\csname PY@tok@nd\endcsname{\def\PY@tc##1{\textcolor[rgb]{0.67,0.13,1.00}{##1}}}
\expandafter\def\csname PY@tok@s\endcsname{\def\PY@tc##1{\textcolor[rgb]{0.73,0.13,0.13}{##1}}}
\expandafter\def\csname PY@tok@sd\endcsname{\let\PY@it=\textit\def\PY@tc##1{\textcolor[rgb]{0.73,0.13,0.13}{##1}}}
\expandafter\def\csname PY@tok@si\endcsname{\let\PY@bf=\textbf\def\PY@tc##1{\textcolor[rgb]{0.73,0.40,0.53}{##1}}}
\expandafter\def\csname PY@tok@se\endcsname{\let\PY@bf=\textbf\def\PY@tc##1{\textcolor[rgb]{0.73,0.40,0.13}{##1}}}
\expandafter\def\csname PY@tok@sr\endcsname{\def\PY@tc##1{\textcolor[rgb]{0.73,0.40,0.53}{##1}}}
\expandafter\def\csname PY@tok@ss\endcsname{\def\PY@tc##1{\textcolor[rgb]{0.10,0.09,0.49}{##1}}}
\expandafter\def\csname PY@tok@sx\endcsname{\def\PY@tc##1{\textcolor[rgb]{0.00,0.50,0.00}{##1}}}
\expandafter\def\csname PY@tok@m\endcsname{\def\PY@tc##1{\textcolor[rgb]{0.40,0.40,0.40}{##1}}}
\expandafter\def\csname PY@tok@gh\endcsname{\let\PY@bf=\textbf\def\PY@tc##1{\textcolor[rgb]{0.00,0.00,0.50}{##1}}}
\expandafter\def\csname PY@tok@gu\endcsname{\let\PY@bf=\textbf\def\PY@tc##1{\textcolor[rgb]{0.50,0.00,0.50}{##1}}}
\expandafter\def\csname PY@tok@gd\endcsname{\def\PY@tc##1{\textcolor[rgb]{0.63,0.00,0.00}{##1}}}
\expandafter\def\csname PY@tok@gi\endcsname{\def\PY@tc##1{\textcolor[rgb]{0.00,0.63,0.00}{##1}}}
\expandafter\def\csname PY@tok@gr\endcsname{\def\PY@tc##1{\textcolor[rgb]{1.00,0.00,0.00}{##1}}}
\expandafter\def\csname PY@tok@ge\endcsname{\let\PY@it=\textit}
\expandafter\def\csname PY@tok@gs\endcsname{\let\PY@bf=\textbf}
\expandafter\def\csname PY@tok@gp\endcsname{\let\PY@bf=\textbf\def\PY@tc##1{\textcolor[rgb]{0.00,0.00,0.50}{##1}}}
\expandafter\def\csname PY@tok@go\endcsname{\def\PY@tc##1{\textcolor[rgb]{0.53,0.53,0.53}{##1}}}
\expandafter\def\csname PY@tok@gt\endcsname{\def\PY@tc##1{\textcolor[rgb]{0.00,0.27,0.87}{##1}}}
\expandafter\def\csname PY@tok@err\endcsname{\def\PY@bc##1{\setlength{\fboxsep}{0pt}\fcolorbox[rgb]{1.00,0.00,0.00}{1,1,1}{\strut ##1}}}
\expandafter\def\csname PY@tok@kc\endcsname{\let\PY@bf=\textbf\def\PY@tc##1{\textcolor[rgb]{0.00,0.50,0.00}{##1}}}
\expandafter\def\csname PY@tok@kd\endcsname{\let\PY@bf=\textbf\def\PY@tc##1{\textcolor[rgb]{0.00,0.50,0.00}{##1}}}
\expandafter\def\csname PY@tok@kn\endcsname{\let\PY@bf=\textbf\def\PY@tc##1{\textcolor[rgb]{0.00,0.50,0.00}{##1}}}
\expandafter\def\csname PY@tok@kr\endcsname{\let\PY@bf=\textbf\def\PY@tc##1{\textcolor[rgb]{0.00,0.50,0.00}{##1}}}
\expandafter\def\csname PY@tok@bp\endcsname{\def\PY@tc##1{\textcolor[rgb]{0.00,0.50,0.00}{##1}}}
\expandafter\def\csname PY@tok@fm\endcsname{\def\PY@tc##1{\textcolor[rgb]{0.00,0.00,1.00}{##1}}}
\expandafter\def\csname PY@tok@vc\endcsname{\def\PY@tc##1{\textcolor[rgb]{0.10,0.09,0.49}{##1}}}
\expandafter\def\csname PY@tok@vg\endcsname{\def\PY@tc##1{\textcolor[rgb]{0.10,0.09,0.49}{##1}}}
\expandafter\def\csname PY@tok@vi\endcsname{\def\PY@tc##1{\textcolor[rgb]{0.10,0.09,0.49}{##1}}}
\expandafter\def\csname PY@tok@vm\endcsname{\def\PY@tc##1{\textcolor[rgb]{0.10,0.09,0.49}{##1}}}
\expandafter\def\csname PY@tok@sa\endcsname{\def\PY@tc##1{\textcolor[rgb]{0.73,0.13,0.13}{##1}}}
\expandafter\def\csname PY@tok@sb\endcsname{\def\PY@tc##1{\textcolor[rgb]{0.73,0.13,0.13}{##1}}}
\expandafter\def\csname PY@tok@sc\endcsname{\def\PY@tc##1{\textcolor[rgb]{0.73,0.13,0.13}{##1}}}
\expandafter\def\csname PY@tok@dl\endcsname{\def\PY@tc##1{\textcolor[rgb]{0.73,0.13,0.13}{##1}}}
\expandafter\def\csname PY@tok@s2\endcsname{\def\PY@tc##1{\textcolor[rgb]{0.73,0.13,0.13}{##1}}}
\expandafter\def\csname PY@tok@sh\endcsname{\def\PY@tc##1{\textcolor[rgb]{0.73,0.13,0.13}{##1}}}
\expandafter\def\csname PY@tok@s1\endcsname{\def\PY@tc##1{\textcolor[rgb]{0.73,0.13,0.13}{##1}}}
\expandafter\def\csname PY@tok@mb\endcsname{\def\PY@tc##1{\textcolor[rgb]{0.40,0.40,0.40}{##1}}}
\expandafter\def\csname PY@tok@mf\endcsname{\def\PY@tc##1{\textcolor[rgb]{0.40,0.40,0.40}{##1}}}
\expandafter\def\csname PY@tok@mh\endcsname{\def\PY@tc##1{\textcolor[rgb]{0.40,0.40,0.40}{##1}}}
\expandafter\def\csname PY@tok@mi\endcsname{\def\PY@tc##1{\textcolor[rgb]{0.40,0.40,0.40}{##1}}}
\expandafter\def\csname PY@tok@il\endcsname{\def\PY@tc##1{\textcolor[rgb]{0.40,0.40,0.40}{##1}}}
\expandafter\def\csname PY@tok@mo\endcsname{\def\PY@tc##1{\textcolor[rgb]{0.40,0.40,0.40}{##1}}}
\expandafter\def\csname PY@tok@ch\endcsname{\let\PY@it=\textit\def\PY@tc##1{\textcolor[rgb]{0.25,0.50,0.50}{##1}}}
\expandafter\def\csname PY@tok@cm\endcsname{\let\PY@it=\textit\def\PY@tc##1{\textcolor[rgb]{0.25,0.50,0.50}{##1}}}
\expandafter\def\csname PY@tok@cpf\endcsname{\let\PY@it=\textit\def\PY@tc##1{\textcolor[rgb]{0.25,0.50,0.50}{##1}}}
\expandafter\def\csname PY@tok@c1\endcsname{\let\PY@it=\textit\def\PY@tc##1{\textcolor[rgb]{0.25,0.50,0.50}{##1}}}
\expandafter\def\csname PY@tok@cs\endcsname{\let\PY@it=\textit\def\PY@tc##1{\textcolor[rgb]{0.25,0.50,0.50}{##1}}}

\def\PYZbs{\char`\\}
\def\PYZus{\char`\_}
\def\PYZob{\char`\{}
\def\PYZcb{\char`\}}
\def\PYZca{\char`\^}
\def\PYZam{\char`\&}
\def\PYZlt{\char`\<}
\def\PYZgt{\char`\>}
\def\PYZsh{\char`\#}
\def\PYZpc{\char`\%}
\def\PYZdl{\char`\$}
\def\PYZhy{\char`\-}
\def\PYZsq{\char`\'}
\def\PYZdq{\char`\"}
\def\PYZti{\char`\~}
% for compatibility with earlier versions
\def\PYZat{@}
\def\PYZlb{[}
\def\PYZrb{]}
\makeatother


    % Exact colors from NB
    \definecolor{incolor}{rgb}{0.0, 0.0, 0.5}
    \definecolor{outcolor}{rgb}{0.545, 0.0, 0.0}



    
    % Prevent overflowing lines due to hard-to-break entities
    \sloppy 
    % Setup hyperref package
    \hypersetup{
      breaklinks=true,  % so long urls are correctly broken across lines
      colorlinks=true,
      urlcolor=urlcolor,
      linkcolor=linkcolor,
      citecolor=citecolor,
      }
    % Slightly bigger margins than the latex defaults
    
    \geometry{verbose,tmargin=1in,bmargin=1in,lmargin=1in,rmargin=1in}
    
    

    \begin{document}
    
    
    \maketitle
    
    

    
    \section{131 Final Project}\label{final-project}

\subsection{Dataset: Crypto Currency from
Kaggle}\label{dataset-crypto-currency-from-kaggle}

\subsubsection{By Carolanne Link and Vileena
Koneru}\label{by-carolanne-link-and-vileena-koneru}

    \paragraph{Import the CSV}\label{import-the-csv}

    \begin{Verbatim}[commandchars=\\\{\}]
{\color{incolor}In [{\color{incolor}228}]:} \PY{k+kn}{import} \PY{n+nn}{numpy} \PY{k}{as} \PY{n+nn}{np}
          \PY{k+kn}{import} \PY{n+nn}{pandas} \PY{k}{as} \PY{n+nn}{pd}
          \PY{k+kn}{import} \PY{n+nn}{matplotlib}\PY{n+nn}{.}\PY{n+nn}{pyplot} \PY{k}{as} \PY{n+nn}{plt}
          \PY{k+kn}{import} \PY{n+nn}{pylab}
          \PY{k+kn}{import} \PY{n+nn}{scipy}\PY{n+nn}{.}\PY{n+nn}{stats} \PY{k}{as} \PY{n+nn}{stats}
          \PY{k+kn}{import} \PY{n+nn}{statsmodels}\PY{n+nn}{.}\PY{n+nn}{api} \PY{k}{as} \PY{n+nn}{sm}
          \PY{k+kn}{from} \PY{n+nn}{sklearn} \PY{k}{import} \PY{n}{linear\PYZus{}model}
          \PY{k+kn}{from} \PY{n+nn}{sklearn}\PY{n+nn}{.}\PY{n+nn}{model\PYZus{}selection} \PY{k}{import} \PY{n}{train\PYZus{}test\PYZus{}split}
          \PY{k+kn}{from} \PY{n+nn}{sklearn}\PY{n+nn}{.}\PY{n+nn}{model\PYZus{}selection} \PY{k}{import} \PY{n}{cross\PYZus{}val\PYZus{}score}
          \PY{k+kn}{import} \PY{n+nn}{seaborn} \PY{k}{as} \PY{n+nn}{sns}
          \PY{n}{cc} \PY{o}{=}\PY{n}{pd}\PY{o}{.}\PY{n}{read\PYZus{}csv}\PY{p}{(}\PY{l+s+s1}{\PYZsq{}}\PY{l+s+s1}{\PYZti{}/Documents/UCLA/Academics/2018 Spring/Stats 131/crypto\PYZhy{}markets.csv}\PY{l+s+s1}{\PYZsq{}}\PY{p}{)}
\end{Verbatim}


    \paragraph{View Head}\label{view-head}

    \begin{Verbatim}[commandchars=\\\{\}]
{\color{incolor}In [{\color{incolor}229}]:} \PY{n+nb}{print}\PY{p}{(}\PY{n}{cc}\PY{o}{.}\PY{n}{head}\PY{p}{(}\PY{p}{)}\PY{p}{)}
\end{Verbatim}


    \begin{Verbatim}[commandchars=\\\{\}]
      slug symbol     name        date  ranknow    open    high     low  \textbackslash{}
0  bitcoin    BTC  Bitcoin  2013-04-28        1  135.30  135.98  132.10   
1  bitcoin    BTC  Bitcoin  2013-04-29        1  134.44  147.49  134.00   
2  bitcoin    BTC  Bitcoin  2013-04-30        1  144.00  146.93  134.05   
3  bitcoin    BTC  Bitcoin  2013-05-01        1  139.00  139.89  107.72   
4  bitcoin    BTC  Bitcoin  2013-05-02        1  116.38  125.60   92.28   

    close  volume        market  close\_ratio  spread  
0  134.21     0.0  1.500520e+09       0.5438    3.88  
1  144.54     0.0  1.491160e+09       0.7813   13.49  
2  139.00     0.0  1.597780e+09       0.3843   12.88  
3  116.99     0.0  1.542820e+09       0.2882   32.17  
4  105.21     0.0  1.292190e+09       0.3881   33.32  

    \end{Verbatim}

    \subsection{Context and Description of the
Data}\label{context-and-description-of-the-data}

    \paragraph{Background information on the subject and field of
study}\label{background-information-on-the-subject-and-field-of-study}

    This dataset came from Kaggle. The user that it belongs to is listed as
Jesse Vent or "jvent". In the About Me section of his profile it says
"Senior data analyst and machine learning enthusiast, working on large
scale public service ICT projects". It also says he's from Adelaide,
Australia and that his preferred languages are R "and a little python".

I don't know much about crypto-currency and therefore I was intrigued to
explore the dataset. Therefore, I did some background research.

For starters, cryptocurrencies in Merriam-Webster Dictionary is defined
as "any form of currency that only exists digitally, that usually has no
central issuing or regulating authority but instead uses a decentralized
system to record transactions and manage the issuance of new units, and
that relies on cryptography to preven counterfeiting and fradulent
transactions". Merriam-Webster also provides an example, "Virtual
currency bitcoin hit the mainstream in 2014. Bitcoin ATMs started
springing up all over the world \ldots{} , allowing people to exchange
cash for the cryptocurrency, a secure digital payment outside of
conventional financial institutions. ---Brenda Poppy"

Wikipedia describes it as "digita asset designed to work as a medium of
exchange that uses strong cryptography to secure financial transactions,
control the creation of additional units, and verify the transfer of
assets".

These definitions are quite close and from what I understand its a
digital currency that is massively unregulated. Bitcoin, which was first
released as open-source software in 2009, is the most well-known.
Wikipedia says since then over 4,000 alterative coins have been created.

Ultimately, I'm very curious to explore this dataset and learn more
about how the cryptocurrency market works and what trends are present.

Cryptocurrency is enabled by the use of Blockchain, a decentralized
record of transactions. A transaction is complete when it is permanently
and unalterably added to the blockchain. Of the various types of
cryptocurrency, Bitcoin is the most popular coin. Cryptocurrency is
exciting to many because a decentralized system would obviate the need
for a central clearing authority, such as a bank. Suggested applications
of this technology are in voting electronically, keeping medical
patients' information secure, managing fractional ownership of
automobiles, financial services and more.

However, cryptocurrency is a highly volatile form of currency with many
financial experts claiming that it has no intrinsic value and is
unsuitable for investments, citing that it is merely a speculative
market. The Bitcoin crash of December 17, 2017 followed speculation that
it would reach \$142,000 - when it in fact fell steadily and reached
\$6500 this year. It is for this reason and the potential that people
see in bitcoin that it is useful to predict where bitcoin will go in
terms of future value. Speculative market or not, being able to predict
the price of cryptocurrenncy is of advantage.

Aside from the name and slug of the coin, this dataset is comprised of
mostly numeric variables with the exception being the date. We can
attempt predictions of the future value as well as observe past trends.

    \subsubsection{Information about Data
Collection}\label{information-about-data-collection}

    \subparagraph{Who Collected the Data}\label{who-collected-the-data}

    The Kaggle user who posted and manages the data gives credit to the
website CoinMarketCap for providing the data.

    \paragraph{When the data was
collected}\label{when-the-data-was-collected}

    The data has no clear collection date. From the context, the collection
seems to be ongoing and Jesse manages it periodically for this dataset.

However, Jesse updates the versions regularly and the first version
\textbf{(Version 1)} was uploaded 10 month ago.

\textbf{Version 2} was uploaded 9 months ago with the note "Historical
cryptocurrency market data and token prices for all tokens: 486006 rows
of data, 1104 different crypto currencies, and Refreshed data to include
54 new coins up until: 10 September 2017".

\textbf{Version 3} was uploaded 9 months ago with the note "Updated data
up until 19th September 2017".

\textbf{Version 4} was uploaded 8 months ago with the note
"Cryptocurrency data includes up to 30/09/2017".

\textbf{Version 5} was uploaded 8 months ago with the note "Fixed
symbols/name and no longer rounding fields to 5 decimals".

\textbf{Version 6} was uploaded 7 months ago with the note "Updates to
handle names better and include current rank".

\textbf{Version 7} was uploaded 6 months ago with the note "Huge
overhaul of formatting and conversions".

\textbf{Version 8} was uploaded 5 months ago with the note "Fixed
duplicates, new variables, refreshed data".

\textbf{Version 9} was uploaded 5 months ago with the note "Updated Data
as of 11/01/2017".

\textbf{Version 10} was uploaded 4 months ago with the note "Refreshed
as of 28 Jan".

\textbf{Version 11} was uploaded 4 months ago with the note "updated
data".

\textbf{Version 12} was uploaded 4 months ago with the note "Updated
data as of 22 Feb 2018".

\textbf{Version 13} was uploaded 2 months ago with the note "Data
current as of 27/03/2018".

\textbf{Version 14} was uploaded 18 days ago with the note "latest
update 21 May".

\textbf{Version 15} was uploaded 18 days ago with the note "Data
refreshed".

\textbf{Version 16} was uploaded 18 days ago with the note "Data refresh
080618".

Upon looking at the date column, we see that the data was collected in
real-time from April 2013 to June 2018.

    \paragraph{How the data was collected}\label{how-the-data-was-collected}

    It is not clearly stated how the data was collected. Neither Jesse Vent
or the MarketCoinCap website specifies collection methods. From the
context and the notes it is clear that MarketCoinCap is a daily updated
source of the cryptocurrency statistics. From the context of the
variables, we know that data is taken every day at both opening and
closing of the market. Over the course of each day the high and low are
also recorded. This means that there is constant monitoring and data
collection of the values throughout the day. They are probably published
at the close of each day recapping the day's changes. Jesse Vent scrapes
this data on what seems to be a pretty regular basis to keep this
dataset up-to-date and properly organized.

    \paragraph{Any implications this may have on
analysis}\label{any-implications-this-may-have-on-analysis}

    Since this is real data coming from real cryptocurrency market values,
the data is very applicable to everyday life. For those who invest in
this, the trends of this data could be a very important part of their
decisions and investment choices.

    \subsection{Exploratory Analysis of the
Data}\label{exploratory-analysis-of-the-data}

    Let's look at the info for the whole dataset.

    \begin{Verbatim}[commandchars=\\\{\}]
{\color{incolor}In [{\color{incolor}230}]:} \PY{n}{cc}\PY{o}{.}\PY{n}{info}\PY{p}{(}\PY{p}{)}
\end{Verbatim}


    \begin{Verbatim}[commandchars=\\\{\}]
<class 'pandas.core.frame.DataFrame'>
RangeIndex: 785024 entries, 0 to 785023
Data columns (total 13 columns):
slug           785024 non-null object
symbol         785024 non-null object
name           785024 non-null object
date           785024 non-null object
ranknow        785024 non-null int64
open           785024 non-null float64
high           785024 non-null float64
low            785024 non-null float64
close          785024 non-null float64
volume         785024 non-null float64
market         785024 non-null float64
close\_ratio    785024 non-null float64
spread         785024 non-null float64
dtypes: float64(8), int64(1), object(4)
memory usage: 77.9+ MB

    \end{Verbatim}

    Let's first explore each variable and its distribution. How many unique
cryptocurrencies are featured here? It claims "all" are featured.

    \begin{Verbatim}[commandchars=\\\{\}]
{\color{incolor}In [{\color{incolor}231}]:} \PY{n}{cc}\PY{p}{[}\PY{l+s+s1}{\PYZsq{}}\PY{l+s+s1}{name}\PY{l+s+s1}{\PYZsq{}}\PY{p}{]}\PY{o}{.}\PY{n}{describe}\PY{p}{(}\PY{p}{)}
\end{Verbatim}


\begin{Verbatim}[commandchars=\\\{\}]
{\color{outcolor}Out[{\color{outcolor}231}]:} count       785024
          unique        1643
          top       HempCoin
          freq          2221
          Name: name, dtype: object
\end{Verbatim}
            
    There are 1643 unique names of currencies with the most frequently
represented currency being HempCoin.

    \begin{Verbatim}[commandchars=\\\{\}]
{\color{incolor}In [{\color{incolor}232}]:} \PY{n+nb}{print}\PY{p}{(}\PY{n}{cc}\PY{p}{[}\PY{l+s+s1}{\PYZsq{}}\PY{l+s+s1}{slug}\PY{l+s+s1}{\PYZsq{}}\PY{p}{]}\PY{o}{.}\PY{n}{describe}\PY{p}{(}\PY{p}{)}\PY{p}{)}
          \PY{n+nb}{print}\PY{p}{(}\PY{n}{cc}\PY{p}{[}\PY{l+s+s1}{\PYZsq{}}\PY{l+s+s1}{symbol}\PY{l+s+s1}{\PYZsq{}}\PY{p}{]}\PY{o}{.}\PY{n}{describe}\PY{p}{(}\PY{p}{)}\PY{p}{)}
\end{Verbatim}


    \begin{Verbatim}[commandchars=\\\{\}]
count       785024
unique        1644
top       namecoin
freq          1866
Name: slug, dtype: object
count     785024
unique      1605
top          NET
freq        1955
Name: symbol, dtype: object

    \end{Verbatim}

    An article from the New York Times explains, "The term slug derives from
the days of hot-metal printing, when printers set type by hand in a
small form called a stick. Later huge Linotype machines turned molten
lead into casts of letters, lines, sentences and paragraphs. A line of
lead in both eras was known as a slug." In news articles, they became an
abbreviation for topics. They could be acronyms or just abbreviated
words. The NY Times gave a few examples, BRIT for Britain and SCOTUS for
Supreme Court of the United States.

In this case, a slug is the short name for a cryptocurrency, there are
1644 slugs for 1643 different coins (name), which means that there is 1
more slug than there are names. The most frequently featured slug is
bitcoin.

    \begin{Verbatim}[commandchars=\\\{\}]
{\color{incolor}In [{\color{incolor}233}]:} \PY{n+nb}{print}\PY{p}{(}\PY{n}{cc}\PY{p}{[}\PY{l+s+s1}{\PYZsq{}}\PY{l+s+s1}{name}\PY{l+s+s1}{\PYZsq{}}\PY{p}{]}\PY{o}{.}\PY{n}{describe}\PY{p}{(}\PY{p}{)}\PY{p}{)}
\end{Verbatim}


    \begin{Verbatim}[commandchars=\\\{\}]
count       785024
unique        1643
top       HempCoin
freq          2221
Name: name, dtype: object

    \end{Verbatim}

    There are 1605 different symbols, which means that some symbols are used
for multiple slugs and names. The top most frequented symbol is NET.

    Next, we'll convert the date column to a datetime object and look at the
date column.

    \begin{Verbatim}[commandchars=\\\{\}]
{\color{incolor}In [{\color{incolor}234}]:} \PY{n}{cc}\PY{p}{[}\PY{l+s+s1}{\PYZsq{}}\PY{l+s+s1}{date}\PY{l+s+s1}{\PYZsq{}}\PY{p}{]} \PY{o}{=}  \PY{n}{pd}\PY{o}{.}\PY{n}{to\PYZus{}datetime}\PY{p}{(}\PY{n}{cc}\PY{p}{[}\PY{l+s+s1}{\PYZsq{}}\PY{l+s+s1}{date}\PY{l+s+s1}{\PYZsq{}}\PY{p}{]}\PY{p}{,} \PY{n+nb}{format}\PY{o}{=}\PY{l+s+s1}{\PYZsq{}}\PY{l+s+s1}{\PYZpc{}}\PY{l+s+s1}{Y\PYZhy{}}\PY{l+s+s1}{\PYZpc{}}\PY{l+s+s1}{m\PYZhy{}}\PY{l+s+si}{\PYZpc{}d}\PY{l+s+s1}{\PYZsq{}}\PY{p}{)}
          \PY{n}{cc}\PY{o}{.}\PY{n}{date}\PY{o}{.}\PY{n}{describe}\PY{p}{(}\PY{p}{)}
\end{Verbatim}


\begin{Verbatim}[commandchars=\\\{\}]
{\color{outcolor}Out[{\color{outcolor}234}]:} count                  785024
          unique                   1866
          top       2018-05-24 00:00:00
          freq                     1599
          first     2013-04-28 00:00:00
          last      2018-06-06 00:00:00
          Name: date, dtype: object
\end{Verbatim}
            
    The date column ranges from April 28, 2013 to June 6, 2018. The top most
frequent date is May 24, 2018.

    Next,we look at the ranknow column.

    \begin{Verbatim}[commandchars=\\\{\}]
{\color{incolor}In [{\color{incolor}235}]:} \PY{n}{rank} \PY{o}{=} \PY{n}{cc}\PY{o}{.}\PY{n}{groupby}\PY{p}{(}\PY{p}{[}\PY{l+s+s1}{\PYZsq{}}\PY{l+s+s1}{ranknow}\PY{l+s+s1}{\PYZsq{}}\PY{p}{,} \PY{l+s+s1}{\PYZsq{}}\PY{l+s+s1}{slug}\PY{l+s+s1}{\PYZsq{}}\PY{p}{]}\PY{p}{)}\PY{p}{[}\PY{l+s+s1}{\PYZsq{}}\PY{l+s+s1}{market}\PY{l+s+s1}{\PYZsq{}}\PY{p}{]}\PY{o}{.}\PY{n}{mean}\PY{p}{(}\PY{p}{)}
          \PY{n}{rank}
\end{Verbatim}


\begin{Verbatim}[commandchars=\\\{\}]
{\color{outcolor}Out[{\color{outcolor}235}]:} ranknow  slug               
          1        bitcoin                3.078214e+10
          2        ethereum               1.892582e+10
          3        ripple                 5.168853e+09
          4        bitcoin-cash           1.845137e+10
          5        eos                    4.189334e+09
          6        litecoin               1.429485e+09
          7        stellar                9.100688e+08
          8        cardano                7.105550e+09
          9        iota                   4.006688e+09
          10       tron                   2.587292e+09
          11       neo                    1.951882e+09
          12       monero                 7.244444e+08
          13       dash                   8.933909e+08
          14       tether                 3.723972e+08
          15       nem                    1.167381e+09
          16       vechain                1.146229e+09
          17       binance-coin           7.443559e+08
          18       ethereum-classic       1.176534e+09
          19       ontology               5.543627e+08
          20       qtum                   1.320165e+09
          21       omisego                1.143071e+09
          22       bytecoin-bcn           1.784239e+08
          23       icon                   1.259400e+09
          24       zilliqa                5.340809e+08
          25       zcash                  5.742421e+08
          26       lisk                   5.489260e+08
          27       aeternity              2.851382e+08
          28       bitcoin-gold           2.057248e+09
          29       decred                 1.655788e+08
          30       0x                     3.752628e+08
                                              {\ldots}     
          1615     deltacredits           1.369835e+03
          1616     x2                     0.000000e+00
          1617     operand                0.000000e+00
          1618     darklisk               0.000000e+00
          1619     richcoin               0.000000e+00
          1620     prismchain             0.000000e+00
          1621     pokecoin               0.000000e+00
          1622     todaycoin              0.000000e+00
          1623     sportscoin             0.000000e+00
          1624     omicron                0.000000e+00
          1625     royalcoin              0.000000e+00
          1626     lepen                  0.000000e+00
          1627     kashhcoin              0.000000e+00
          1628     ur                     0.000000e+00
          1629     fazzcoin               0.000000e+00
          1630     bitok                  0.000000e+00
          1631     aseancoin              0.000000e+00
          1632     ox-fina                0.000000e+00
          1633     sigmacoin              0.000000e+00
          1634     dutch-coin             0.000000e+00
          1635     fapcoin                0.000000e+00
          1636     plexcoin               0.000000e+00
          1637     madcoin                3.050317e+05
          1638     titanium-blockchain    0.000000e+00
          1639     vcash                  3.384252e+06
          1640     ereal                  9.210886e+04
          1641     davorcoin              0.000000e+00
          1643     entcash                0.000000e+00
          1644     jingtum-tech           0.000000e+00
          1645     aston                  0.000000e+00
          Name: market, Length: 1644, dtype: float64
\end{Verbatim}
            
    From looking at the MarketCoinCap website,we see that the currencies are
ranked based upon "Market Cap" value. According to Investopedia, "market
cap" or Market Capitalization is the total dollar market value of a
company's outstanding shares...it is calculated by multiplying a
company's shares outstanding by the current market price of one share.
The investment community uses this figure to determine a company's size,
as opposed to using sales or total asset figures. Using market
capitalization to show the size of a company is important because
company size is a basic determinant of various characteristics in which
investors are interested, including risk." It also provides the example
calculation of a company with 20 million shares selling at 100 dollars a
share would have a market cap of 2 billion dollars.\\
Therefore, it comes as no surprise that bitcoin is ranked \#1, it is the
most popular crypto-currency.

    \begin{Verbatim}[commandchars=\\\{\}]
{\color{incolor}In [{\color{incolor}236}]:} \PY{n+nb}{print}\PY{p}{(}\PY{n}{cc}\PY{p}{[}\PY{l+s+s1}{\PYZsq{}}\PY{l+s+s1}{ranknow}\PY{l+s+s1}{\PYZsq{}}\PY{p}{]}\PY{o}{.}\PY{n}{describe}\PY{p}{(}\PY{p}{)}\PY{p}{)}
\end{Verbatim}


    \begin{Verbatim}[commandchars=\\\{\}]
count    785024.000000
mean        842.650876
std         452.624872
min           1.000000
25\%         472.000000
50\%         910.000000
75\%        1185.000000
max        1645.000000
Name: ranknow, dtype: float64

    \end{Verbatim}

    The ranknow column has a minimum of 1 and a maximum of 1645.

    \begin{Verbatim}[commandchars=\\\{\}]
{\color{incolor}In [{\color{incolor}237}]:} \PY{n+nb}{print}\PY{p}{(}\PY{n}{cc}\PY{p}{[}\PY{l+s+s1}{\PYZsq{}}\PY{l+s+s1}{open}\PY{l+s+s1}{\PYZsq{}}\PY{p}{]}\PY{o}{.}\PY{n}{describe}\PY{p}{(}\PY{p}{)}\PY{p}{)}
          \PY{n+nb}{print}\PY{p}{(}\PY{n}{cc}\PY{p}{[}\PY{l+s+s1}{\PYZsq{}}\PY{l+s+s1}{spread}\PY{l+s+s1}{\PYZsq{}}\PY{p}{]}\PY{o}{.}\PY{n}{describe}\PY{p}{(}\PY{p}{)}\PY{p}{)}
\end{Verbatim}


    \begin{Verbatim}[commandchars=\\\{\}]
count    7.850240e+05
mean     3.550859e+02
std      1.403939e+04
min      2.500000e-09
25\%      1.117000e-03
50\%      1.723150e-02
75\%      2.159187e-01
max      2.298390e+06
Name: open, dtype: float64
count    7.850240e+05
mean     1.252368e+02
std      7.379191e+03
min      0.000000e+00
25\%      0.000000e+00
50\%      0.000000e+00
75\%      4.000000e-02
max      1.770563e+06
Name: spread, dtype: float64

    \end{Verbatim}

    From the summary we see opening prices are highly skewed to the right.
Open prices are the opening value of each coin at the beginning of the
market's day. The maximum is 2298390 and the minimum is close to 0
(2.5x10\^{}-9) , with a mean of 355.0859 and a huge standard deviation
of 14039.39. The 50\% quantile is at 0.01723150.

    The spread, according to Investopedia is "the difference between the bid
and the ask price of a security or asset. It can also refer to an
options position established by purchasing one option and selling
another option of the same class but of a different series." Jesse Vent
defines this column as "Spread is the \$USD difference between the high
and low values for the day". For our data, the spread is also highly
skewed to the right. The minimum is 0 and the mximum is 1770563, with a
mean of 125.2368 and a standard deviation of 7379.191. The 50\% quantile
is 0.

    To get a better idea of the distribution, we took the log.

    \begin{Verbatim}[commandchars=\\\{\}]
{\color{incolor}In [{\color{incolor}238}]:} \PY{n}{cc}\PY{p}{[}\PY{l+s+s1}{\PYZsq{}}\PY{l+s+s1}{logopen}\PY{l+s+s1}{\PYZsq{}}\PY{p}{]} \PY{o}{=} \PY{n}{np}\PY{o}{.}\PY{n}{log}\PY{p}{(}\PY{n}{cc}\PY{p}{[}\PY{l+s+s1}{\PYZsq{}}\PY{l+s+s1}{open}\PY{l+s+s1}{\PYZsq{}}\PY{p}{]}\PY{p}{)}
          \PY{n}{plt}\PY{o}{.}\PY{n}{hist}\PY{p}{(}\PY{n}{cc}\PY{p}{[}\PY{l+s+s1}{\PYZsq{}}\PY{l+s+s1}{logopen}\PY{l+s+s1}{\PYZsq{}}\PY{p}{]}\PY{p}{)}
          \PY{n}{plt}\PY{o}{.}\PY{n}{show}\PY{p}{(}\PY{p}{)}
\end{Verbatim}


    \begin{center}
    \adjustimage{max size={0.9\linewidth}{0.9\paperheight}}{output_39_0.png}
    \end{center}
    { \hspace*{\fill} \\}
    
    Plotting a histogram of the log-transformed values created a plot that
resembles a normal distribution.

    We created a log transformed plot for the spread as well.

    \begin{Verbatim}[commandchars=\\\{\}]
{\color{incolor}In [{\color{incolor}239}]:} \PY{n}{cc}\PY{p}{[}\PY{l+s+s1}{\PYZsq{}}\PY{l+s+s1}{logspread}\PY{l+s+s1}{\PYZsq{}}\PY{p}{]} \PY{o}{=} \PY{n}{np}\PY{o}{.}\PY{n}{log1p}\PY{p}{(}\PY{n}{cc}\PY{p}{[}\PY{l+s+s1}{\PYZsq{}}\PY{l+s+s1}{spread}\PY{l+s+s1}{\PYZsq{}}\PY{p}{]}\PY{p}{)}
          \PY{n}{plt}\PY{o}{.}\PY{n}{hist}\PY{p}{(}\PY{n}{cc}\PY{p}{[}\PY{l+s+s1}{\PYZsq{}}\PY{l+s+s1}{logspread}\PY{l+s+s1}{\PYZsq{}}\PY{p}{]}\PY{p}{)}
          \PY{n}{plt}\PY{o}{.}\PY{n}{show}\PY{p}{(}\PY{p}{)}
\end{Verbatim}


    \begin{center}
    \adjustimage{max size={0.9\linewidth}{0.9\paperheight}}{output_42_0.png}
    \end{center}
    { \hspace*{\fill} \\}
    
    Adding 1 to the 0 values, log-transformed spread is still highly skewed.

    Next, we look at the summary statistics for the high value. This is the
highest value that crypto-currency attains durring the market day. We're
starting to expect all the values to be skewed to the right.

    \begin{Verbatim}[commandchars=\\\{\}]
{\color{incolor}In [{\color{incolor}240}]:} \PY{n}{cc}\PY{o}{.}\PY{n}{high}\PY{o}{.}\PY{n}{describe}\PY{p}{(}\PY{p}{)}
\end{Verbatim}


\begin{Verbatim}[commandchars=\\\{\}]
{\color{outcolor}Out[{\color{outcolor}240}]:} count    7.850240e+05
          mean     4.233219e+02
          std      1.733459e+04
          min      3.200000e-09
          25\%      1.305000e-03
          50\%      1.980350e-02
          75\%      2.424478e-01
          max      2.926100e+06
          Name: high, dtype: float64
\end{Verbatim}
            
    The high values are skewed to the right, as expected. The minimum is
close to 0 at (3.2x10\^{}-9) and a maximum of 2926100, with a mean of
423.3219 and a standard deviation of 17334.59. The 50\% quantile is
0.01980350. We took the log transformation to plot a histogram of the
high values.

    \begin{Verbatim}[commandchars=\\\{\}]
{\color{incolor}In [{\color{incolor}241}]:} \PY{n}{cc}\PY{p}{[}\PY{l+s+s1}{\PYZsq{}}\PY{l+s+s1}{loghigh}\PY{l+s+s1}{\PYZsq{}}\PY{p}{]}\PY{o}{=}\PY{n}{np}\PY{o}{.}\PY{n}{log}\PY{p}{(}\PY{n}{cc}\PY{p}{[}\PY{l+s+s1}{\PYZsq{}}\PY{l+s+s1}{high}\PY{l+s+s1}{\PYZsq{}}\PY{p}{]}\PY{p}{)}
          \PY{n}{plt}\PY{o}{.}\PY{n}{hist}\PY{p}{(}\PY{n}{cc}\PY{p}{[}\PY{l+s+s1}{\PYZsq{}}\PY{l+s+s1}{loghigh}\PY{l+s+s1}{\PYZsq{}}\PY{p}{]}\PY{p}{)}
          \PY{n}{plt}\PY{o}{.}\PY{n}{show}\PY{p}{(}\PY{p}{)}
\end{Verbatim}


    \begin{center}
    \adjustimage{max size={0.9\linewidth}{0.9\paperheight}}{output_47_0.png}
    \end{center}
    { \hspace*{\fill} \\}
    
    These results look the same as those for the opening values, the log
transformation allows for normal behavior.

    Next, we look at the low values. As the name would imply, this is the
lowest value that the crypto-currency attains during the market day.

    \begin{Verbatim}[commandchars=\\\{\}]
{\color{incolor}In [{\color{incolor}242}]:} \PY{n}{cc}\PY{o}{.}\PY{n}{low}\PY{o}{.}\PY{n}{describe}\PY{p}{(}\PY{p}{)}
\end{Verbatim}


\begin{Verbatim}[commandchars=\\\{\}]
{\color{outcolor}Out[{\color{outcolor}242}]:} count    7.850240e+05
          mean     2.980848e+02
          std      1.157055e+04
          min      9.200000e-14
          25\%      9.630000e-04
          50\%      1.511100e-02
          75\%      1.915900e-01
          max      2.030590e+06
          Name: low, dtype: float64
\end{Verbatim}
            
    The low points span between 0 and 2030590. The mean is 298.0848 with a
standard The span is hugely right skewed with 75\% of the values being
below 0.19159.

    \begin{Verbatim}[commandchars=\\\{\}]
{\color{incolor}In [{\color{incolor}243}]:} \PY{n}{cc}\PY{p}{[}\PY{l+s+s1}{\PYZsq{}}\PY{l+s+s1}{loglow}\PY{l+s+s1}{\PYZsq{}}\PY{p}{]} \PY{o}{=} \PY{n}{np}\PY{o}{.}\PY{n}{log}\PY{p}{(}\PY{n}{cc}\PY{p}{[}\PY{l+s+s1}{\PYZsq{}}\PY{l+s+s1}{low}\PY{l+s+s1}{\PYZsq{}}\PY{p}{]}\PY{p}{)}
          \PY{n}{plt}\PY{o}{.}\PY{n}{hist}\PY{p}{(}\PY{n}{cc}\PY{p}{[}\PY{l+s+s1}{\PYZsq{}}\PY{l+s+s1}{loglow}\PY{l+s+s1}{\PYZsq{}}\PY{p}{]}\PY{p}{)}
          \PY{n}{plt}\PY{o}{.}\PY{n}{show}\PY{p}{(}\PY{p}{)}
\end{Verbatim}


    \begin{center}
    \adjustimage{max size={0.9\linewidth}{0.9\paperheight}}{output_52_0.png}
    \end{center}
    { \hspace*{\fill} \\}
    
    The log transform makes the data look somewhat normal, but it doesn't
appear as normal as some of the other variables did under this
transformation.

    Next, we look at the close variable. Again, as the name would imply,
this is the value at the close of the market day.

    \begin{Verbatim}[commandchars=\\\{\}]
{\color{incolor}In [{\color{incolor}244}]:} \PY{n}{cc}\PY{o}{.}\PY{n}{close}\PY{o}{.}\PY{n}{describe}\PY{p}{(}\PY{p}{)}
\end{Verbatim}


\begin{Verbatim}[commandchars=\\\{\}]
{\color{outcolor}Out[{\color{outcolor}244}]:} count    7.850240e+05
          mean     3.536499e+02
          std      1.396510e+04
          min      0.000000e+00
          25\%      1.119000e-03
          50\%      1.723000e-02
          75\%      2.156592e-01
          max      2.300740e+06
          Name: close, dtype: float64
\end{Verbatim}
            
    The close values are between 0 and 2399740. The mean is 353.6499 with a
standard deviation of 13965.10. The 50\% quantile is at 0.01723.

We log transform the close values.

    \begin{Verbatim}[commandchars=\\\{\}]
{\color{incolor}In [{\color{incolor}245}]:} \PY{n}{cc}\PY{p}{[}\PY{l+s+s1}{\PYZsq{}}\PY{l+s+s1}{logclose}\PY{l+s+s1}{\PYZsq{}}\PY{p}{]}\PY{o}{=}\PY{n}{np}\PY{o}{.}\PY{n}{log1p}\PY{p}{(}\PY{n}{cc}\PY{p}{[}\PY{l+s+s1}{\PYZsq{}}\PY{l+s+s1}{close}\PY{l+s+s1}{\PYZsq{}}\PY{p}{]}\PY{p}{)}
          \PY{n}{plt}\PY{o}{.}\PY{n}{hist}\PY{p}{(}\PY{n}{cc}\PY{p}{[}\PY{l+s+s1}{\PYZsq{}}\PY{l+s+s1}{logclose}\PY{l+s+s1}{\PYZsq{}}\PY{p}{]}\PY{p}{)}
          \PY{n}{plt}\PY{o}{.}\PY{n}{show}\PY{p}{(}\PY{p}{)}
\end{Verbatim}


    \begin{center}
    \adjustimage{max size={0.9\linewidth}{0.9\paperheight}}{output_57_0.png}
    \end{center}
    { \hspace*{\fill} \\}
    
    Closing value is similar to spread value, extremely skewed, or maybe a
different transform is needed.

    We then look at the Volume variable.

    \begin{Verbatim}[commandchars=\\\{\}]
{\color{incolor}In [{\color{incolor}246}]:} \PY{n+nb}{print}\PY{p}{(}\PY{n}{cc}\PY{o}{.}\PY{n}{volume}\PY{o}{.}\PY{n}{describe}\PY{p}{(}\PY{p}{)}\PY{p}{)}
\end{Verbatim}


    \begin{Verbatim}[commandchars=\\\{\}]
count    7.850240e+05
mean     7.459260e+06
std      1.817992e+08
min      0.000000e+00
25\%      6.400000e+01
50\%      1.201000e+03
75\%      3.963225e+04
max      2.384090e+10
Name: volume, dtype: float64

    \end{Verbatim}

    The volume variable shows a low of 0, a high of 23840900000. The mean is
7459260 with a standard deviaton of 181799200. The 50\% quantile is
1201.

    We look further into log transforming this variable, like we did with
the others.

    \begin{Verbatim}[commandchars=\\\{\}]
{\color{incolor}In [{\color{incolor}247}]:} \PY{n}{cc}\PY{p}{[}\PY{l+s+s1}{\PYZsq{}}\PY{l+s+s1}{logvol}\PY{l+s+s1}{\PYZsq{}}\PY{p}{]}\PY{o}{=}\PY{n}{np}\PY{o}{.}\PY{n}{log1p}\PY{p}{(}\PY{n}{cc}\PY{p}{[}\PY{l+s+s1}{\PYZsq{}}\PY{l+s+s1}{volume}\PY{l+s+s1}{\PYZsq{}}\PY{p}{]}\PY{p}{)}
          \PY{n}{plt}\PY{o}{.}\PY{n}{hist}\PY{p}{(}\PY{n}{cc}\PY{p}{[}\PY{l+s+s1}{\PYZsq{}}\PY{l+s+s1}{logvol}\PY{l+s+s1}{\PYZsq{}}\PY{p}{]}\PY{p}{)}
          \PY{n}{plt}\PY{o}{.}\PY{n}{xlabel}\PY{p}{(}\PY{l+s+s2}{\PYZdq{}}\PY{l+s+s2}{Log Volume}\PY{l+s+s2}{\PYZdq{}}\PY{p}{)}
          \PY{n}{plt}\PY{o}{.}\PY{n}{ylabel}\PY{p}{(}\PY{l+s+s2}{\PYZdq{}}\PY{l+s+s2}{Freq}\PY{l+s+s2}{\PYZdq{}}\PY{p}{)}
          \PY{n}{plt}\PY{o}{.}\PY{n}{show}\PY{p}{(}\PY{p}{)}
          \PY{n}{cc}\PY{o}{.}\PY{n}{logvol}\PY{o}{.}\PY{n}{describe}\PY{p}{(}\PY{p}{)}
\end{Verbatim}


    \begin{center}
    \adjustimage{max size={0.9\linewidth}{0.9\paperheight}}{output_63_0.png}
    \end{center}
    { \hspace*{\fill} \\}
    
\begin{Verbatim}[commandchars=\\\{\}]
{\color{outcolor}Out[{\color{outcolor}247}]:} count    785024.000000
          mean          7.459935
          std           4.414825
          min           0.000000
          25\%           4.174387
          50\%           7.091742
          75\%          10.587424
          max          23.894668
          Name: logvol, dtype: float64
\end{Verbatim}
            
    Volume: Half of all individual coins are exchanged somewhere between
e\^{}4 and e\^{}11 times.

    Next, we look into market value. We discussed this value briefly when we
looked at the rank variable.

    \begin{Verbatim}[commandchars=\\\{\}]
{\color{incolor}In [{\color{incolor}248}]:} \PY{n}{cc}\PY{o}{.}\PY{n}{market}\PY{o}{.}\PY{n}{describe}\PY{p}{(}\PY{p}{)}
\end{Verbatim}


\begin{Verbatim}[commandchars=\\\{\}]
{\color{outcolor}Out[{\color{outcolor}248}]:} count    7.850240e+05
          mean     1.563100e+08
          std      3.478147e+09
          min      0.000000e+00
          25\%      1.294600e+04
          50\%      1.932045e+05
          75\%      3.635550e+06
          max      3.261410e+11
          Name: market, dtype: float64
\end{Verbatim}
            
    The minimum is 0 and the maximum 326141000000. The mean is 156310000 and
the standard deviation is 3478147000. The 50\% quantile is 193204.5. We
see market value is skewed to the right.

Again, we choose to try a log transformation.

    \begin{Verbatim}[commandchars=\\\{\}]
{\color{incolor}In [{\color{incolor}249}]:} \PY{n}{cc}\PY{p}{[}\PY{l+s+s1}{\PYZsq{}}\PY{l+s+s1}{logmarket}\PY{l+s+s1}{\PYZsq{}}\PY{p}{]} \PY{o}{=} \PY{n}{np}\PY{o}{.}\PY{n}{log1p}\PY{p}{(}\PY{n}{cc}\PY{p}{[}\PY{l+s+s1}{\PYZsq{}}\PY{l+s+s1}{market}\PY{l+s+s1}{\PYZsq{}}\PY{p}{]}\PY{p}{)}
          \PY{n}{plt}\PY{o}{.}\PY{n}{hist}\PY{p}{(}\PY{n}{cc}\PY{p}{[}\PY{l+s+s1}{\PYZsq{}}\PY{l+s+s1}{logmarket}\PY{l+s+s1}{\PYZsq{}}\PY{p}{]}\PY{p}{)}
          \PY{n}{plt}\PY{o}{.}\PY{n}{show}\PY{p}{(}\PY{p}{)}
\end{Verbatim}


    \begin{center}
    \adjustimage{max size={0.9\linewidth}{0.9\paperheight}}{output_68_0.png}
    \end{center}
    { \hspace*{\fill} \\}
    
    Log transform of market gives an almost-normal distribution.

    Lastly, we'll look at the closing ratio. Close ratio is the daily close
rate, min-maxed with the high and low values for the day. Close Ratio =
(Close-Low)/(High-Low)

    \begin{Verbatim}[commandchars=\\\{\}]
{\color{incolor}In [{\color{incolor}250}]:} \PY{n}{cc}\PY{o}{.}\PY{n}{close\PYZus{}ratio}\PY{o}{.}\PY{n}{describe}\PY{p}{(}\PY{p}{)}
\end{Verbatim}


\begin{Verbatim}[commandchars=\\\{\}]
{\color{outcolor}Out[{\color{outcolor}250}]:} count    7.850240e+05
          mean             -inf
          std               NaN
          min              -inf
          25\%      1.552000e-01
          50\%      4.342000e-01
          75\%      7.586000e-01
          max      1.000000e+00
          Name: close\_ratio, dtype: float64
\end{Verbatim}
            
    \begin{Verbatim}[commandchars=\\\{\}]
{\color{incolor}In [{\color{incolor}251}]:} \PY{n}{plt}\PY{o}{.}\PY{n}{boxplot}\PY{p}{(}\PY{n}{cc}\PY{o}{.}\PY{n}{close\PYZus{}ratio}\PY{p}{)}
          \PY{n}{plt}\PY{o}{.}\PY{n}{title}\PY{p}{(}\PY{l+s+s2}{\PYZdq{}}\PY{l+s+s2}{Close Ratio}\PY{l+s+s2}{\PYZdq{}}\PY{p}{)}
          \PY{n}{plt}\PY{o}{.}\PY{n}{show}\PY{p}{(}\PY{p}{)}
\end{Verbatim}


    \begin{center}
    \adjustimage{max size={0.9\linewidth}{0.9\paperheight}}{output_72_0.png}
    \end{center}
    { \hspace*{\fill} \\}
    
    We see that since this is a ratio, the minimum and mean come across as
-inf, and the standard deviation is NaN. The 25\% quantile is 0.1552.
The 50\% quantile is at 0.4342. The 75\% quantile is 0.7586. The maximum
is, of course, 1. Close ratio is slightly skewed. We can't log-transform
this variable.

    Let's look at our new dataset with more accommodating log-transformed
variables and drop the others.

    \begin{Verbatim}[commandchars=\\\{\}]
{\color{incolor}In [{\color{incolor}252}]:} \PY{n}{cc}\PY{o}{.}\PY{n}{drop}\PY{p}{(}\PY{n}{columns}\PY{o}{=}\PY{p}{[}\PY{l+s+s1}{\PYZsq{}}\PY{l+s+s1}{logspread}\PY{l+s+s1}{\PYZsq{}}\PY{p}{,}\PY{l+s+s1}{\PYZsq{}}\PY{l+s+s1}{logclose}\PY{l+s+s1}{\PYZsq{}}\PY{p}{]}\PY{p}{)}
          \PY{n}{cc}\PY{o}{.}\PY{n}{head}\PY{p}{(}\PY{p}{)}
\end{Verbatim}


\begin{Verbatim}[commandchars=\\\{\}]
{\color{outcolor}Out[{\color{outcolor}252}]:}       slug symbol     name       date  ranknow    open    high     low  \textbackslash{}
          0  bitcoin    BTC  Bitcoin 2013-04-28        1  135.30  135.98  132.10   
          1  bitcoin    BTC  Bitcoin 2013-04-29        1  134.44  147.49  134.00   
          2  bitcoin    BTC  Bitcoin 2013-04-30        1  144.00  146.93  134.05   
          3  bitcoin    BTC  Bitcoin 2013-05-01        1  139.00  139.89  107.72   
          4  bitcoin    BTC  Bitcoin 2013-05-02        1  116.38  125.60   92.28   
          
              close  volume        market  close\_ratio  spread   logopen  logspread  \textbackslash{}
          0  134.21     0.0  1.500520e+09       0.5438    3.88  4.907495   1.585145   
          1  144.54     0.0  1.491160e+09       0.7813   13.49  4.901118   2.673459   
          2  139.00     0.0  1.597780e+09       0.3843   12.88  4.969813   2.630449   
          3  116.99     0.0  1.542820e+09       0.2882   32.17  4.934474   3.501646   
          4  105.21     0.0  1.292190e+09       0.3881   33.32  4.756861   3.535728   
          
              loghigh    loglow  logclose  logvol  logmarket  
          0  4.912508  4.883559  4.906829     0.0  21.129078  
          1  4.993760  4.897840  4.980451     0.0  21.122820  
          2  4.989956  4.898213  4.941642     0.0  21.191881  
          3  4.940856  4.679535  4.770600     0.0  21.156878  
          4  4.833102  4.524827  4.665418     0.0  20.979604  
\end{Verbatim}
            
    We are going to explore the relationships between some variables with
plots to give us ideas of what to model.

    \begin{Verbatim}[commandchars=\\\{\}]
{\color{incolor}In [{\color{incolor}253}]:} \PY{n}{plt}\PY{o}{.}\PY{n}{scatter}\PY{p}{(}\PY{n}{cc}\PY{p}{[}\PY{l+s+s1}{\PYZsq{}}\PY{l+s+s1}{open}\PY{l+s+s1}{\PYZsq{}}\PY{p}{]}\PY{p}{,}\PY{n}{cc}\PY{p}{[}\PY{l+s+s1}{\PYZsq{}}\PY{l+s+s1}{high}\PY{l+s+s1}{\PYZsq{}}\PY{p}{]}\PY{p}{)}
          \PY{n}{plt}\PY{o}{.}\PY{n}{title}\PY{p}{(}\PY{l+s+s2}{\PYZdq{}}\PY{l+s+s2}{Opening vs High Value}\PY{l+s+s2}{\PYZdq{}}\PY{p}{)} \PY{c+c1}{\PYZsh{}Change x and y axis scale}
          \PY{n}{plt}\PY{o}{.}\PY{n}{show}\PY{p}{(}\PY{p}{)}
\end{Verbatim}


    \begin{center}
    \adjustimage{max size={0.9\linewidth}{0.9\paperheight}}{output_77_0.png}
    \end{center}
    { \hspace*{\fill} \\}
    
    A scatterplot showing the relationship between the opening value of a
coin and it's highest value. Opening value and high are directly
correlated.

    \begin{Verbatim}[commandchars=\\\{\}]
{\color{incolor}In [{\color{incolor}254}]:} \PY{n}{plt}\PY{o}{.}\PY{n}{scatter}\PY{p}{(}\PY{n}{cc}\PY{p}{[}\PY{l+s+s1}{\PYZsq{}}\PY{l+s+s1}{high}\PY{l+s+s1}{\PYZsq{}}\PY{p}{]}\PY{p}{,}\PY{n}{cc}\PY{p}{[}\PY{l+s+s1}{\PYZsq{}}\PY{l+s+s1}{spread}\PY{l+s+s1}{\PYZsq{}}\PY{p}{]}\PY{p}{)}
          \PY{n}{plt}\PY{o}{.}\PY{n}{title}\PY{p}{(}\PY{l+s+s2}{\PYZdq{}}\PY{l+s+s2}{High vs Spread Price}\PY{l+s+s2}{\PYZdq{}}\PY{p}{)}
          \PY{n}{plt}\PY{o}{.}\PY{n}{xlabel}\PY{p}{(}\PY{l+s+s2}{\PYZdq{}}\PY{l+s+s2}{High Value}\PY{l+s+s2}{\PYZdq{}}\PY{p}{)} \PY{c+c1}{\PYZsh{}Change x and y axis scale}
          \PY{n}{plt}\PY{o}{.}\PY{n}{ylabel}\PY{p}{(}\PY{l+s+s2}{\PYZdq{}}\PY{l+s+s2}{Spread}\PY{l+s+s2}{\PYZdq{}}\PY{p}{)}
          \PY{n}{plt}\PY{o}{.}\PY{n}{show}\PY{p}{(}\PY{p}{)}
\end{Verbatim}


    \begin{center}
    \adjustimage{max size={0.9\linewidth}{0.9\paperheight}}{output_79_0.png}
    \end{center}
    { \hspace*{\fill} \\}
    
    A scatterplot showing highest value and spread. We can see a fan in the
variance, non-constant variance is obvious.

    Let's look at which currencies have the highest high and close values.

    \begin{Verbatim}[commandchars=\\\{\}]
{\color{incolor}In [{\color{incolor}255}]:} \PY{n}{highmeans} \PY{o}{=} \PY{n}{cc}\PY{o}{.}\PY{n}{groupby}\PY{p}{(}\PY{p}{[}\PY{l+s+s1}{\PYZsq{}}\PY{l+s+s1}{name}\PY{l+s+s1}{\PYZsq{}}\PY{p}{]}\PY{p}{)}\PY{p}{[}\PY{l+s+s1}{\PYZsq{}}\PY{l+s+s1}{high}\PY{l+s+s1}{\PYZsq{}}\PY{p}{]}\PY{o}{.}\PY{n}{mean}\PY{p}{(}\PY{p}{)}
\end{Verbatim}


    \begin{Verbatim}[commandchars=\\\{\}]
{\color{incolor}In [{\color{incolor}256}]:} \PY{n}{highmeans}\PY{o}{.}\PY{n}{sort\PYZus{}values}\PY{p}{(}\PY{n}{ascending} \PY{o}{=} \PY{k+kc}{False}\PY{p}{)}
\end{Verbatim}


\begin{Verbatim}[commandchars=\\\{\}]
{\color{outcolor}Out[{\color{outcolor}256}]:} name
          Bit20                 573800.507742
          Project-X             291122.594070
          42-coin                17652.879396
          Russian Miner Coin     11995.447078
          Primalbase Token        4615.546571
          IDEX Membership         4304.785652
          bitBTC                  3113.846189
          CryptopiaFeeShares      2673.788523
          Bitcoin                 1951.660102
          bitGold                 1461.007459
          Internet of Things      1261.868184
          Bitcoin Cash            1182.859091
          Mixin                    977.271409
          Maker                    890.052639
          WETH                     847.635417
          300 Token                556.053070
          Byteball Bytes           345.830228
          Zcash                    264.443959
          TerraNova                253.844006
          SegWit2x                 215.186281
          Ethereum                 205.027354
          Lightning Bitcoin        197.714129
          Sovereign Hero           187.035625
          Veritaseum               181.205357
          BT2 [CST]                171.860237
          Bitcoin Gold             166.013568
          Gnosis                   161.364403
          Dash                     121.775981
          Xaurum                   109.507864
          United Bitcoin           100.948304
                                    {\ldots}      
          IncaKoin                   0.000093
          Elite                      0.000091
          PopularCoin                0.000087
          Coupecoin                  0.000084
          VapersCoin                 0.000084
          Carboncoin                 0.000083
          EmberCoin                  0.000082
          Protean                    0.000073
          Pandacoin                  0.000067
          SatoshiMadness             0.000059
          InflationCoin              0.000054
          LiteDoge                   0.000054
          EXRNchain                  0.000052
          Powercoin                  0.000050
          Selfiecoin                 0.000047
          Zeitcoin                   0.000042
          Infinitecoin               0.000040
          StrongHands                0.000038
          Mooncoin                   0.000036
          SmileyCoin                 0.000029
          Slothcoin                  0.000028
          Karmacoin                  0.000025
          NewYorkCoin                0.000023
          Photon                     0.000017
          FedoraCoin                 0.000012
          RabbitCoin                 0.000011
          Dimecoin                   0.000010
          BunnyCoin                  0.000009
          GCN Coin                   0.000007
          The Cypherfunks            0.000006
          Name: high, Length: 1643, dtype: float64
\end{Verbatim}
            
    \begin{Verbatim}[commandchars=\\\{\}]
{\color{incolor}In [{\color{incolor}257}]:} \PY{n}{closemeans} \PY{o}{=} \PY{n}{cc}\PY{o}{.}\PY{n}{groupby}\PY{p}{(}\PY{p}{[}\PY{l+s+s1}{\PYZsq{}}\PY{l+s+s1}{name}\PY{l+s+s1}{\PYZsq{}}\PY{p}{]}\PY{p}{)}\PY{p}{[}\PY{l+s+s1}{\PYZsq{}}\PY{l+s+s1}{close}\PY{l+s+s1}{\PYZsq{}}\PY{p}{]}\PY{o}{.}\PY{n}{mean}\PY{p}{(}\PY{p}{)}
\end{Verbatim}


    \begin{Verbatim}[commandchars=\\\{\}]
{\color{incolor}In [{\color{incolor}258}]:} \PY{n}{closemeans}\PY{o}{.}\PY{n}{sort\PYZus{}values}\PY{p}{(}\PY{n}{ascending} \PY{o}{=} \PY{k+kc}{False}\PY{p}{)}
\end{Verbatim}


\begin{Verbatim}[commandchars=\\\{\}]
{\color{outcolor}Out[{\color{outcolor}258}]:} name
          Bit20                 485437.176774
          Project-X             231403.399461
          42-coin                15309.871509
          Russian Miner Coin     10827.388037
          Primalbase Token        4239.073048
          IDEX Membership         3936.608406
          bitBTC                  2840.327591
          CryptopiaFeeShares      2416.614765
          Bitcoin                 1889.513762
          bitGold                 1366.551230
          Internet of Things      1134.038905
          Bitcoin Cash            1104.780376
          Mixin                    895.261275
          Maker                    839.815556
          WETH                     719.216806
          300 Token                480.605222
          Byteball Bytes           318.646717
          Zcash                    237.645051
          Ethereum                 196.889603
          SegWit2x                 181.911308
          Sovereign Hero           173.063406
          TerraNova                170.493456
          Lightning Bitcoin        165.915419
          Veritaseum               155.713325
          Gnosis                   151.362836
          Bitcoin Gold             150.840793
          BT2 [CST]                146.461706
          Dash                     116.109724
          Xaurum                   102.320777
          United Bitcoin            87.891287
                                    {\ldots}      
          Elite                      0.000075
          PopularCoin                0.000070
          EmberCoin                  0.000065
          IncaKoin                   0.000065
          Carboncoin                 0.000064
          Coupecoin                  0.000061
          VapersCoin                 0.000058
          Pandacoin                  0.000055
          SatoshiMadness             0.000051
          Protean                    0.000051
          EXRNchain                  0.000045
          LiteDoge                   0.000042
          Powercoin                  0.000041
          InflationCoin              0.000040
          Selfiecoin                 0.000038
          Infinitecoin               0.000035
          Zeitcoin                   0.000032
          StrongHands                0.000029
          Mooncoin                   0.000028
          Karmacoin                  0.000023
          SmileyCoin                 0.000023
          NewYorkCoin                0.000019
          Slothcoin                  0.000016
          Photon                     0.000013
          FedoraCoin                 0.000010
          Dimecoin                   0.000008
          RabbitCoin                 0.000008
          BunnyCoin                  0.000007
          GCN Coin                   0.000005
          The Cypherfunks            0.000005
          Name: close, Length: 1643, dtype: float64
\end{Verbatim}
            
    In both cases of high and close, the top value is from Bit20 and the
lowest is The Cypherfunks.

    \begin{Verbatim}[commandchars=\\\{\}]
{\color{incolor}In [{\color{incolor}259}]:} \PY{n}{volmeans} \PY{o}{=} \PY{n}{cc}\PY{o}{.}\PY{n}{groupby}\PY{p}{(}\PY{p}{[}\PY{l+s+s1}{\PYZsq{}}\PY{l+s+s1}{name}\PY{l+s+s1}{\PYZsq{}}\PY{p}{]}\PY{p}{)}\PY{p}{[}\PY{l+s+s1}{\PYZsq{}}\PY{l+s+s1}{volume}\PY{l+s+s1}{\PYZsq{}}\PY{p}{]}\PY{o}{.}\PY{n}{mean}\PY{p}{(}\PY{p}{)}
          \PY{n}{volmeans}\PY{o}{.}\PY{n}{sort\PYZus{}values}\PY{p}{(}\PY{n}{ascending} \PY{o}{=} \PY{k+kc}{False}\PY{p}{)}
\end{Verbatim}


\begin{Verbatim}[commandchars=\\\{\}]
{\color{outcolor}Out[{\color{outcolor}259}]:} name
          Bitcoin             1.171548e+09
          Bitcoin Cash        9.292763e+08
          Ethereum            7.075375e+08
          EOS                 5.363772e+08
          Tether              4.634491e+08
          TRON                3.354088e+08
          Cardano             2.162328e+08
          Qtum                2.144078e+08
          Ripple              1.719680e+08
          Ethereum Classic    1.584033e+08
          CK USD              1.388396e+08
          Litecoin            1.208760e+08
          Huobi Token         1.087217e+08
          IOTA                9.974698e+07
          Bitcoin Gold        9.727490e+07
          Mithril             9.703662e+07
          Ontology            9.427183e+07
          NEO                 8.797341e+07
          IOST                7.076920e+07
          aelf                6.680524e+07
          OmiseGO             6.648238e+07
          True Chain          6.619169e+07
          Cortex              6.610622e+07
          Status              6.456758e+07
          Binance Coin        6.300307e+07
          Hshare              6.140922e+07
          ICON                6.091282e+07
          Storm               6.027209e+07
          QuarkChain          5.910860e+07
          VeChain             5.848345e+07
                                  {\ldots}     
          Bitz                2.164561e+02
          PrismChain          2.120461e+02
          Shilling            2.111863e+02
          ImpulseCoin         2.042978e+02
          RichCoin            2.040903e+02
          PosEx               2.003247e+02
          Opescoin            1.957782e+02
          Steps               1.957478e+02
          NodeCoin            1.900261e+02
          Polcoin             1.765823e+02
          EggCoin             1.639026e+02
          BowsCoin            1.616835e+02
          Selfiecoin          1.612840e+02
          Metal Music Coin    1.599432e+02
          PX                  1.585195e+02
          Cycling Coin        1.575436e+02
          BTCtalkcoin         1.455293e+02
          Californium         1.370664e+02
          DAPPSTER            1.356612e+02
          Firecoin            1.199340e+02
          PLNcoin             1.170874e+02
          MindCoin            1.133014e+02
          iBank               1.126857e+02
          MetalCoin           1.084583e+02
          PokeCoin            1.006804e+02
          DarkLisk            9.478374e+01
          Bitcoin 21          9.372652e+01
          SportsCoin          9.240208e+01
          TAGRcoin            8.508323e+01
          X2                  8.047327e+01
          Name: volume, Length: 1643, dtype: float64
\end{Verbatim}
            
    Bit20 has both the highest high and highest close value. The lowest high
belongs to The Cypherfunks and lowest close value belongs to X2.

    \begin{Verbatim}[commandchars=\\\{\}]
{\color{incolor}In [{\color{incolor}260}]:} \PY{n}{cc}\PY{o}{.}\PY{n}{volume}\PY{o}{.}\PY{n}{describe}\PY{p}{(}\PY{p}{)}
\end{Verbatim}


\begin{Verbatim}[commandchars=\\\{\}]
{\color{outcolor}Out[{\color{outcolor}260}]:} count    7.850240e+05
          mean     7.459260e+06
          std      1.817992e+08
          min      0.000000e+00
          25\%      6.400000e+01
          50\%      1.201000e+03
          75\%      3.963225e+04
          max      2.384090e+10
          Name: volume, dtype: float64
\end{Verbatim}
            
    The largest volume on average is Bitcoin, which comes as no surprise
since it's the most well-known of the cryptocurrencies.

    Volume is the amount of that coin traded in that day in the last 24 hrs.
A demand indicator for future price of coin. More demand, higher future
price. Mean volume is 7459260.

    Next, we look at the relationship between open and close prices.

    \begin{Verbatim}[commandchars=\\\{\}]
{\color{incolor}In [{\color{incolor}261}]:} \PY{n}{plt}\PY{o}{.}\PY{n}{scatter}\PY{p}{(}\PY{n}{cc}\PY{p}{[}\PY{l+s+s1}{\PYZsq{}}\PY{l+s+s1}{open}\PY{l+s+s1}{\PYZsq{}}\PY{p}{]}\PY{p}{,}\PY{n}{cc}\PY{p}{[}\PY{l+s+s1}{\PYZsq{}}\PY{l+s+s1}{close}\PY{l+s+s1}{\PYZsq{}}\PY{p}{]}\PY{p}{)}
          \PY{n}{plt}\PY{o}{.}\PY{n}{title}\PY{p}{(}\PY{l+s+s2}{\PYZdq{}}\PY{l+s+s2}{Open vs Close Price}\PY{l+s+s2}{\PYZdq{}}\PY{p}{)}
          \PY{n}{plt}\PY{o}{.}\PY{n}{xlabel}\PY{p}{(}\PY{l+s+s2}{\PYZdq{}}\PY{l+s+s2}{Open Value}\PY{l+s+s2}{\PYZdq{}}\PY{p}{)} \PY{c+c1}{\PYZsh{}Change x and y axis scale}
          \PY{n}{plt}\PY{o}{.}\PY{n}{ylabel}\PY{p}{(}\PY{l+s+s2}{\PYZdq{}}\PY{l+s+s2}{Close}\PY{l+s+s2}{\PYZdq{}}\PY{p}{)}
          \PY{n}{mod\PYZus{}fit} \PY{o}{=} \PY{n}{sm}\PY{o}{.}\PY{n}{OLS}\PY{p}{(}\PY{n}{cc}\PY{o}{.}\PY{n}{open}\PY{p}{,} \PY{n}{cc}\PY{o}{.}\PY{n}{close}\PY{p}{)}\PY{o}{.}\PY{n}{fit}\PY{p}{(}\PY{p}{)}
          \PY{n}{res} \PY{o}{=} \PY{n}{mod\PYZus{}fit}\PY{o}{.}\PY{n}{resid} \PY{c+c1}{\PYZsh{} residuals}
          \PY{n}{plt}\PY{o}{.}\PY{n}{show}\PY{p}{(}\PY{p}{)}
\end{Verbatim}


    \begin{center}
    \adjustimage{max size={0.9\linewidth}{0.9\paperheight}}{output_93_0.png}
    \end{center}
    { \hspace*{\fill} \\}
    
    This relationship looks like one that could very accurately be modeled
by a linear model. We will explore this below.

    \subsection{Model}\label{model}

    \subsubsection{Simple Linear Model Between Open and
Close}\label{simple-linear-model-between-open-and-close}

    \begin{Verbatim}[commandchars=\\\{\}]
{\color{incolor}In [{\color{incolor}262}]:} \PY{n}{X} \PY{o}{=} \PY{n}{cc}\PY{o}{.}\PY{n}{open}
          \PY{n}{X}\PY{o}{.}\PY{n}{shape}
          \PY{n}{X} \PY{o}{=} \PY{n}{X}\PY{o}{.}\PY{n}{values}\PY{o}{.}\PY{n}{reshape}\PY{p}{(}\PY{p}{[}\PY{l+m+mi}{785024}\PY{p}{,}\PY{l+m+mi}{1}\PY{p}{]}\PY{p}{)}
          \PY{n+nb}{print}\PY{p}{(}\PY{n}{X}\PY{o}{.}\PY{n}{shape}\PY{p}{)}
          \PY{n}{y}\PY{o}{=} \PY{n}{cc}\PY{o}{.}\PY{n}{close}
\end{Verbatim}


    \begin{Verbatim}[commandchars=\\\{\}]
(785024, 1)

    \end{Verbatim}

    \begin{Verbatim}[commandchars=\\\{\}]
{\color{incolor}In [{\color{incolor}263}]:} \PY{n}{regr} \PY{o}{=} \PY{n}{linear\PYZus{}model}\PY{o}{.}\PY{n}{LinearRegression}\PY{p}{(}\PY{p}{)}
\end{Verbatim}


    \begin{Verbatim}[commandchars=\\\{\}]
{\color{incolor}In [{\color{incolor}264}]:} \PY{n}{regr}\PY{o}{.}\PY{n}{fit}\PY{p}{(}\PY{n}{X}\PY{p}{,}\PY{n}{y}\PY{p}{)}
\end{Verbatim}


\begin{Verbatim}[commandchars=\\\{\}]
{\color{outcolor}Out[{\color{outcolor}264}]:} LinearRegression(copy\_X=True, fit\_intercept=True, n\_jobs=1, normalize=False)
\end{Verbatim}
            
    \subparagraph{Coefficients and
Intercepts}\label{coefficients-and-intercepts}

    \begin{Verbatim}[commandchars=\\\{\}]
{\color{incolor}In [{\color{incolor}265}]:} \PY{n+nb}{print}\PY{p}{(}\PY{l+s+s1}{\PYZsq{}}\PY{l+s+s1}{Coefficients: }\PY{l+s+se}{\PYZbs{}n}\PY{l+s+s1}{\PYZsq{}}\PY{p}{,} \PY{n}{regr}\PY{o}{.}\PY{n}{coef\PYZus{}}\PY{p}{)}
          \PY{n+nb}{print}\PY{p}{(}\PY{l+s+s1}{\PYZsq{}}\PY{l+s+s1}{Intercept: }\PY{l+s+se}{\PYZbs{}n}\PY{l+s+s1}{\PYZsq{}}\PY{p}{,} \PY{n}{regr}\PY{o}{.}\PY{n}{intercept\PYZus{}}\PY{p}{)}
\end{Verbatim}


    \begin{Verbatim}[commandchars=\\\{\}]
Coefficients: 
 [0.9511476]
Intercept: 
 15.910857376516503

    \end{Verbatim}

    \subparagraph{R\^{}2 Value}\label{r2-value}

    \begin{Verbatim}[commandchars=\\\{\}]
{\color{incolor}In [{\color{incolor}266}]:} \PY{c+c1}{\PYZsh{} r squared value}
          \PY{n}{regr}\PY{o}{.}\PY{n}{score}\PY{p}{(}\PY{n}{X}\PY{p}{,} \PY{n}{y}\PY{p}{)}  \PY{c+c1}{\PYZsh{} when we fit all of the data points}
\end{Verbatim}


\begin{Verbatim}[commandchars=\\\{\}]
{\color{outcolor}Out[{\color{outcolor}266}]:} 0.9143320966535049
\end{Verbatim}
            
    This means that this model explains 91.4\% of the variation in the
response variable around its mean.

    \subparagraph{Plot}\label{plot}

    \begin{Verbatim}[commandchars=\\\{\}]
{\color{incolor}In [{\color{incolor}267}]:} \PY{n}{plt}\PY{o}{.}\PY{n}{plot}\PY{p}{(}\PY{n}{X}\PY{p}{,} \PY{n}{regr}\PY{o}{.}\PY{n}{predict}\PY{p}{(}\PY{n}{X}\PY{p}{)}\PY{p}{,}\PY{n}{color}\PY{o}{=}\PY{l+s+s1}{\PYZsq{}}\PY{l+s+s1}{r}\PY{l+s+s1}{\PYZsq{}}\PY{p}{)}
          \PY{n}{plt}\PY{o}{.}\PY{n}{scatter}\PY{p}{(}\PY{n}{cc}\PY{p}{[}\PY{l+s+s1}{\PYZsq{}}\PY{l+s+s1}{open}\PY{l+s+s1}{\PYZsq{}}\PY{p}{]}\PY{p}{,}\PY{n}{cc}\PY{p}{[}\PY{l+s+s1}{\PYZsq{}}\PY{l+s+s1}{close}\PY{l+s+s1}{\PYZsq{}}\PY{p}{]}\PY{p}{)}
          \PY{n}{plt}\PY{o}{.}\PY{n}{title}\PY{p}{(}\PY{l+s+s2}{\PYZdq{}}\PY{l+s+s2}{Open vs Close Price}\PY{l+s+s2}{\PYZdq{}}\PY{p}{)}
          \PY{n}{plt}\PY{o}{.}\PY{n}{xlabel}\PY{p}{(}\PY{l+s+s2}{\PYZdq{}}\PY{l+s+s2}{Open Value}\PY{l+s+s2}{\PYZdq{}}\PY{p}{)} \PY{c+c1}{\PYZsh{}Change x and y axis scale}
          \PY{n}{plt}\PY{o}{.}\PY{n}{ylabel}\PY{p}{(}\PY{l+s+s2}{\PYZdq{}}\PY{l+s+s2}{Close}\PY{l+s+s2}{\PYZdq{}}\PY{p}{)}
          \PY{n}{mod\PYZus{}fit} \PY{o}{=} \PY{n}{sm}\PY{o}{.}\PY{n}{OLS}\PY{p}{(}\PY{n}{cc}\PY{o}{.}\PY{n}{open}\PY{p}{,} \PY{n}{cc}\PY{o}{.}\PY{n}{close}\PY{p}{)}\PY{o}{.}\PY{n}{fit}\PY{p}{(}\PY{p}{)}
          \PY{n}{res} \PY{o}{=} \PY{n}{mod\PYZus{}fit}\PY{o}{.}\PY{n}{resid} \PY{c+c1}{\PYZsh{} residuals}
          \PY{n}{plt}\PY{o}{.}\PY{n}{show}\PY{p}{(}\PY{p}{)}
          \PY{n}{y\PYZus{}pred} \PY{o}{=} \PY{n}{regr}\PY{o}{.}\PY{n}{predict}\PY{p}{(}\PY{n}{X}\PY{p}{)}
\end{Verbatim}


    \begin{center}
    \adjustimage{max size={0.9\linewidth}{0.9\paperheight}}{output_106_0.png}
    \end{center}
    { \hspace*{\fill} \\}
    
    \begin{Verbatim}[commandchars=\\\{\}]
{\color{incolor}In [{\color{incolor}268}]:} \PY{n}{residuals\PYZus{}linear} \PY{o}{=} \PY{n}{y} \PY{o}{\PYZhy{}} \PY{n}{y\PYZus{}pred}
          \PY{n}{sns}\PY{o}{.}\PY{n}{distplot}\PY{p}{(}\PY{n}{residuals\PYZus{}linear}\PY{p}{)}
\end{Verbatim}


\begin{Verbatim}[commandchars=\\\{\}]
{\color{outcolor}Out[{\color{outcolor}268}]:} <matplotlib.axes.\_subplots.AxesSubplot at 0x1c13cb8470>
\end{Verbatim}
            
    \begin{center}
    \adjustimage{max size={0.9\linewidth}{0.9\paperheight}}{output_107_1.png}
    \end{center}
    { \hspace*{\fill} \\}
    
    The data does look normal.

    \begin{Verbatim}[commandchars=\\\{\}]
{\color{incolor}In [{\color{incolor}269}]:} \PY{n}{plt}\PY{o}{.}\PY{n}{scatter}\PY{p}{(}\PY{n}{res}\PY{p}{,} \PY{n}{y\PYZus{}pred}\PY{p}{)}
          \PY{n}{plt}\PY{o}{.}\PY{n}{show}\PY{p}{(}\PY{p}{)}
\end{Verbatim}


    \begin{center}
    \adjustimage{max size={0.9\linewidth}{0.9\paperheight}}{output_109_0.png}
    \end{center}
    { \hspace*{\fill} \\}
    
    There are no obvious trends in the residuals.

    \subparagraph{Cross-Validation}\label{cross-validation}

    \begin{Verbatim}[commandchars=\\\{\}]
{\color{incolor}In [{\color{incolor}270}]:} \PY{n}{cv\PYZus{}results} \PY{o}{=} \PY{n}{cross\PYZus{}val\PYZus{}score}\PY{p}{(}\PY{n}{regr}\PY{p}{,} \PY{n}{X}\PY{p}{,} \PY{n}{y}\PY{p}{,} \PY{n}{cv} \PY{o}{=} \PY{l+m+mi}{5}\PY{p}{)}  \PY{c+c1}{\PYZsh{} 5 fold cross validation}
          
          \PY{n+nb}{print}\PY{p}{(}\PY{n}{cv\PYZus{}results}\PY{p}{)}  \PY{c+c1}{\PYZsh{} lots of variation}
          \PY{c+c1}{\PYZsh{} an undesirable trait}
\end{Verbatim}


    \begin{Verbatim}[commandchars=\\\{\}]
[0.99152461 0.95170542 0.90813792 0.97480055 0.92823925]

    \end{Verbatim}

    The R\^{}2 and CV values reassure us that the simple linear model is a
good fit.

    We will try Ridge regression to get something that hopefully performs a
little bit better.

    \subsubsection{Ridge Regression}\label{ridge-regression}

    \begin{Verbatim}[commandchars=\\\{\}]
{\color{incolor}In [{\color{incolor}271}]:} \PY{n}{ridgemodel} \PY{o}{=} \PY{n}{linear\PYZus{}model}\PY{o}{.}\PY{n}{Ridge}\PY{p}{(}\PY{n}{alpha} \PY{o}{=}\PY{l+m+mi}{1}\PY{p}{)}
          \PY{n}{fit2}\PY{o}{=} \PY{n}{ridgemodel}\PY{o}{.}\PY{n}{fit}\PY{p}{(}\PY{n}{X}\PY{p}{,}\PY{n}{y}\PY{p}{)}
\end{Verbatim}


    \subparagraph{Coefficient and
Intercept}\label{coefficient-and-intercept}

    \begin{Verbatim}[commandchars=\\\{\}]
{\color{incolor}In [{\color{incolor}272}]:} \PY{n+nb}{print}\PY{p}{(}\PY{l+s+s1}{\PYZsq{}}\PY{l+s+s1}{Ridge Coefficient: }\PY{l+s+se}{\PYZbs{}n}\PY{l+s+s1}{\PYZsq{}}\PY{p}{,} \PY{n}{ridgemodel}\PY{o}{.}\PY{n}{coef\PYZus{}}\PY{p}{)}
          \PY{n+nb}{print}\PY{p}{(}\PY{l+s+s1}{\PYZsq{}}\PY{l+s+s1}{Ridge Intercept: }\PY{l+s+se}{\PYZbs{}n}\PY{l+s+s1}{\PYZsq{}}\PY{p}{,} \PY{n}{ridgemodel}\PY{o}{.}\PY{n}{intercept\PYZus{}}\PY{p}{)}
\end{Verbatim}


    \begin{Verbatim}[commandchars=\\\{\}]
Ridge Coefficient: 
 [0.9511476]
Ridge Intercept: 
 15.910857376530089

    \end{Verbatim}

    \subparagraph{Ridge Model Score}\label{ridge-model-score}

    \begin{Verbatim}[commandchars=\\\{\}]
{\color{incolor}In [{\color{incolor}273}]:} \PY{n}{ridgemodel}\PY{o}{.}\PY{n}{score}\PY{p}{(}\PY{n}{X}\PY{p}{,}\PY{n}{y}\PY{p}{)}
\end{Verbatim}


\begin{Verbatim}[commandchars=\\\{\}]
{\color{outcolor}Out[{\color{outcolor}273}]:} 0.9143320966535049
\end{Verbatim}
            
    This means that this model explains 91.4\% of the variation in the
response variable around its mean.

    \subparagraph{Cross-Validation}\label{cross-validation}

    \begin{Verbatim}[commandchars=\\\{\}]
{\color{incolor}In [{\color{incolor}274}]:} \PY{n}{cv\PYZus{}results} \PY{o}{=} \PY{n}{cross\PYZus{}val\PYZus{}score}\PY{p}{(}\PY{n}{ridgemodel}\PY{p}{,} \PY{n}{X}\PY{p}{,} \PY{n}{y}\PY{p}{,} \PY{n}{cv} \PY{o}{=} \PY{l+m+mi}{5}\PY{p}{)}  \PY{c+c1}{\PYZsh{} 5 fold cross validation}
          \PY{n+nb}{print}\PY{p}{(}\PY{n}{cv\PYZus{}results}\PY{p}{)}  
\end{Verbatim}


    \begin{Verbatim}[commandchars=\\\{\}]
[0.99152461 0.95170542 0.90813792 0.97480055 0.92823925]

    \end{Verbatim}

    \subparagraph{Plot}\label{plot}

    \begin{Verbatim}[commandchars=\\\{\}]
{\color{incolor}In [{\color{incolor}275}]:} \PY{n}{plt}\PY{o}{.}\PY{n}{plot}\PY{p}{(}\PY{n}{X}\PY{p}{,} \PY{n}{ridgemodel}\PY{o}{.}\PY{n}{predict}\PY{p}{(}\PY{n}{X}\PY{p}{)}\PY{p}{,}\PY{n}{color}\PY{o}{=}\PY{l+s+s1}{\PYZsq{}}\PY{l+s+s1}{r}\PY{l+s+s1}{\PYZsq{}}\PY{p}{)}
          \PY{n}{plt}\PY{o}{.}\PY{n}{scatter}\PY{p}{(}\PY{n}{cc}\PY{p}{[}\PY{l+s+s1}{\PYZsq{}}\PY{l+s+s1}{open}\PY{l+s+s1}{\PYZsq{}}\PY{p}{]}\PY{p}{,}\PY{n}{cc}\PY{p}{[}\PY{l+s+s1}{\PYZsq{}}\PY{l+s+s1}{close}\PY{l+s+s1}{\PYZsq{}}\PY{p}{]}\PY{p}{)}
          \PY{n}{plt}\PY{o}{.}\PY{n}{title}\PY{p}{(}\PY{l+s+s2}{\PYZdq{}}\PY{l+s+s2}{Open vs Close Price}\PY{l+s+s2}{\PYZdq{}}\PY{p}{)}
          \PY{n}{plt}\PY{o}{.}\PY{n}{xlabel}\PY{p}{(}\PY{l+s+s2}{\PYZdq{}}\PY{l+s+s2}{Open Value}\PY{l+s+s2}{\PYZdq{}}\PY{p}{)} \PY{c+c1}{\PYZsh{}Change x and y axis scale}
          \PY{n}{plt}\PY{o}{.}\PY{n}{ylabel}\PY{p}{(}\PY{l+s+s2}{\PYZdq{}}\PY{l+s+s2}{Close}\PY{l+s+s2}{\PYZdq{}}\PY{p}{)}
          \PY{n}{plt}\PY{o}{.}\PY{n}{show}\PY{p}{(}\PY{p}{)}
          \PY{n}{y\PYZus{}pred} \PY{o}{=} \PY{n}{ridgemodel}\PY{o}{.}\PY{n}{predict}\PY{p}{(}\PY{n}{X}\PY{p}{)}
\end{Verbatim}


    \begin{center}
    \adjustimage{max size={0.9\linewidth}{0.9\paperheight}}{output_125_0.png}
    \end{center}
    { \hspace*{\fill} \\}
    
    We use both Linaer Regression and Ridge Regression to create a moel for
this data. They came out very similarly because it is a simple linear
regression. We chose not to include more predictors because our R\^{}2
and our Ridge scores are very good, both over 91\%.

    \subsubsection{Multiple Linear
Regression}\label{multiple-linear-regression}

    \begin{Verbatim}[commandchars=\\\{\}]
{\color{incolor}In [{\color{incolor}276}]:} \PY{n}{regr2} \PY{o}{=} \PY{n}{linear\PYZus{}model}\PY{o}{.}\PY{n}{LinearRegression}\PY{p}{(}\PY{p}{)}
          \PY{n}{X2} \PY{o}{=} \PY{n}{cc}\PY{p}{[}\PY{p}{[}\PY{l+s+s1}{\PYZsq{}}\PY{l+s+s1}{open}\PY{l+s+s1}{\PYZsq{}}\PY{p}{,}\PY{l+s+s1}{\PYZsq{}}\PY{l+s+s1}{high}\PY{l+s+s1}{\PYZsq{}}\PY{p}{,}\PY{l+s+s1}{\PYZsq{}}\PY{l+s+s1}{low}\PY{l+s+s1}{\PYZsq{}}\PY{p}{]}\PY{p}{]}
\end{Verbatim}


    \begin{Verbatim}[commandchars=\\\{\}]
{\color{incolor}In [{\color{incolor}282}]:} \PY{n}{X2}\PY{o}{.}\PY{n}{shape}
          \PY{n}{X2} \PY{o}{=} \PY{n}{X2}\PY{o}{.}\PY{n}{values}\PY{o}{.}\PY{n}{reshape}\PY{p}{(}\PY{p}{[}\PY{l+m+mi}{785024}\PY{p}{,}\PY{l+m+mi}{3}\PY{p}{]}\PY{p}{)}
          \PY{n+nb}{print}\PY{p}{(}\PY{n}{X2}\PY{o}{.}\PY{n}{shape}\PY{p}{)}
\end{Verbatim}


    \begin{Verbatim}[commandchars=\\\{\}]

        ---------------------------------------------------------------------------

        AttributeError                            Traceback (most recent call last)

        <ipython-input-282-2f1ef4f2c7c2> in <module>()
          1 X2.shape
    ----> 2 X2 = X2.values.reshape([785024,3])
          3 print(X2.shape)


        AttributeError: 'numpy.ndarray' object has no attribute 'values'

    \end{Verbatim}

    \begin{Verbatim}[commandchars=\\\{\}]
{\color{incolor}In [{\color{incolor} }]:} \PY{n}{regr2}\PY{o}{.}\PY{n}{fit}\PY{p}{(}\PY{n}{X2}\PY{p}{,}\PY{n}{y}\PY{p}{)}
\end{Verbatim}


    \subparagraph{Coefficients and
Intercepts}\label{coefficients-and-intercepts}

    \begin{Verbatim}[commandchars=\\\{\}]
{\color{incolor}In [{\color{incolor} }]:} \PY{n+nb}{print}\PY{p}{(}\PY{l+s+s1}{\PYZsq{}}\PY{l+s+s1}{Coefficients: }\PY{l+s+se}{\PYZbs{}n}\PY{l+s+s1}{\PYZsq{}}\PY{p}{,} \PY{n}{regr2}\PY{o}{.}\PY{n}{coef\PYZus{}}\PY{p}{)}
        \PY{n+nb}{print}\PY{p}{(}\PY{l+s+s1}{\PYZsq{}}\PY{l+s+s1}{Intercept: }\PY{l+s+se}{\PYZbs{}n}\PY{l+s+s1}{\PYZsq{}}\PY{p}{,} \PY{n}{regr2}\PY{o}{.}\PY{n}{intercept\PYZus{}}\PY{p}{)}
\end{Verbatim}


    \subparagraph{R\^{}2}\label{r2}

    \begin{Verbatim}[commandchars=\\\{\}]
{\color{incolor}In [{\color{incolor} }]:} \PY{c+c1}{\PYZsh{} r squared value}
        \PY{n}{regr2}\PY{o}{.}\PY{n}{score}\PY{p}{(}\PY{n}{X2}\PY{p}{,} \PY{n}{y}\PY{p}{)}  \PY{c+c1}{\PYZsh{} when we fit all of the data points}
\end{Verbatim}


    \subparagraph{Cross-Validation}\label{cross-validation}

    \begin{Verbatim}[commandchars=\\\{\}]
{\color{incolor}In [{\color{incolor} }]:} \PY{n}{cv\PYZus{}results} \PY{o}{=} \PY{n}{cross\PYZus{}val\PYZus{}score}\PY{p}{(}\PY{n}{regr2}\PY{p}{,} \PY{n}{X2}\PY{p}{,} \PY{n}{y}\PY{p}{,} \PY{n}{cv} \PY{o}{=} \PY{l+m+mi}{5}\PY{p}{)}  \PY{c+c1}{\PYZsh{} 5 fold cross validation}
        \PY{n+nb}{print}\PY{p}{(}\PY{n}{cv\PYZus{}results}\PY{p}{)}  
\end{Verbatim}


    The multiple linear regression, using high and low in addition to the
open predictor gives a R\^{}2 of 97.3\%! This means that this model
explains 97.3\% of the variation in the response variable around its
mean.


    % Add a bibliography block to the postdoc
    
    
    
    \end{document}
